%titolo{L'Emmanuel}
%autore{Mammoli}
%album{}
%tonalita{Mi}
%famiglia{Liturgica}
%gruppo{}
%momenti{}
%identificatore{l_emmanuel}
%data_revisione{2014_09_30}
%trascrittore{Francesco Endrici}
\beginsong{L'Emmanuel}[by={Mammoli}]
\beginverse
Dall'\[E]orizzonte una grande luce
\[B]viaggia nella storia
e \[A]lungo gli anni ha vinto il buio
fa\[B]cendosi Me\[B7]moria.
E il\[E]luminando la nostra vita
\[B]chiaro ci rivela
che \[A]non si vive se non si cerca
\[F#-]la veri\[B7]\[E]\[B]{tà\dots}
\endverse
\beginverse
Da ^mille strade arriviamo a Roma
sui ^passi della fede
sen^tiamo l'eco della Parola
^che risuona an^cora
da ^queste mura, da questo cielo
^per il mondo intero
è ^vivo oggi, è l'Uomo Vero
^Cristo tra ^noi.
\endverse
\beginchorus
Siamo \[E]qui \[A]
sotto la stessa luce \[F#-]
sotto la sua croce \[D]
cantando ad \[B]una \[B7]voce.
\[E]È l'Emmanu\[B]el, l'Emmanu\[A]el,
Emmanu\[B]el. \[B7]
\[E]È l'Emmanu\[B]el, Emmanu\[A]el.
\endchorus
\beginverse
%\chordsoff
Dal^la città di chi ha versato
il ^sangue per amore
ed ^ha cambiato il vecchio mondo
vo^gliamo ripar^tire
se^guendo Cristo, insieme a Pietro
ri^nasce in noi la fede
Pa^rola viva che ci rinnova
^e cresce in ^noi. 
\endverse
\beginchorus
%\chordsoff 
Rit. 
\endchorus
\transpose{1}
\beginverse
Un ^grande dono che Dio che ci ha fatto
è ^Cristo il Suo Figlio
e l'^umanità è rinnovata
^è in lui sal^vata
è ^vero uomo, è vero Dio
è il ^pane della Vita
che ad ^ogni uomo ai suoi fratelli
^ridone^rà.
\endverse
\beginchorus
Siamo \[E]qui \[A]
sotto la stessa luce \[F#-]
sotto la sua croce \[D]
cantando ad \[B]una \[B7]voce.
\[E]È l'Emmanu\[B]el, l'Emmanu\[A]el,
Emmanu\[B]el. \[B7]
\[E]È l'Emmanu\[B]el, Emmanu\[A]el.
\endchorus
\beginverse
%\chordsoff
La ^morte è uccisa, la vita ha vinto
è ^Pasqua in tutto il mondo
un ^vento soffia in ogni uomo
lo ^Spirito fe^condo
che ^porta avanti nella storia
la ^Chiesa sua sposa
sot^to lo sguardo di Maria
^comuni^tà.
\endverse
\beginchorus
%\chordsoff 
Rit. 
\endchorus
\beginverse
%\chordsoff
Noi ^debitori del passato, di ^secoli di storia,
di ^vite date per amore, di ^santi che han cre^duto
di ^uomini che ad alta quota in^segnano a volare,
di c^hi la storia sa cambiare ^come Ge^sù.
\endverse
\beginchorus
%\chordsoff 
Rit. 
\endchorus
\beginverse
%\chordsoff
È ^giunta un'era di primavera,
è ^tempo di cambiare, è ^oggi il giorno sempre
nuovo ^per ricomin^ciare,
per ^dare svolte, parole nuove
e ^convertire il cuore,
per ^dire al mondo, ad ogni uomo, Si^gnore Ge^sù.
\endverse
\beginchorus
%\chordsoff 
Rit. 
\endchorus
\endsong
%titolo{Dio è morto}
%autore{Francesco Guccini}
%album{}
%tonalita{SOL}
%famiglia{altre}
%gruppo{}
%momenti{}
%identificatore{dio_è_morto}
%data_revisione{}
%trascrittore{Francesco Raccanello}
\beginsong{Dio è morto}[
	by={Guccini}]
	
\beginverse
\[G]Ho visto\[D] 
\[D]{la }gente della mia età andare via 
\[B-]lungo le strade che non portano mai a niente, 
\[G]cercare il sogno che conduce alla pazzia 
\[A]nella ricerca di qual\[A7]cosa che non trovano 
nel mondo che hanno \[D]già, 
dentro alle notti che dal vino son bagnate, 
\[B-]dentro alle stanze da pastiglie trasformate, 
\[G]lungo alle nuvole di fumo del mondo fatto di città, 
essere \[A]contro ad ingoiare la \[A7]nostra stanca civiltà 
e un Dio che è \[D]morto, 
ai \[G]bordi delle \[A7]strade Dio è \[D]morto, 
nelle \[G]auto prese a \[A7]rate Dio è \[D]morto, 
nei \[G]miti dell'es\[A7]tate\dots Dio è morto. 
\endverse
\chordsoff
\beginverse
Mi han detto 
che questa mia generazione ormai non crede 
in ciò che spesso han mascherato con la fede, 
nei miti eterni della patria o dell'eroe 
perché è venuto ormai il momento di negare 
tutto ciò che è falsità, le fedi fatte di abitudine e paura, 
una politica che è solo far carriera, 
il perbenismo interessato, la dignità fatta di vuoto, 
l' ipocrisia di chi sta sempre con la ragione e mai col torto 
e un dio che è morto, 
nei campi di sterminio Dio è morto, 
coi miti della razza Dio è morto 
con gli odi di partito Dio è morto. 
\endverse
\beginverse
Ma penso 
che questa mia generazione è preparata 
a un mondo nuovo e a una speranza appena nata, 
ad un futuro che ha già in mano, 
a una rivolta senza armi, 
perché noi tutti ormai sappiamo 
che se Dio muore è per tre giorni e poi risorge, 
in ciò che noi crediamo Dio è risorto, 
in ciò che noi vogliamo Dio è risorto, 
nel mondo che faremo Dio è risorto.
\endverse
\endsong

%titolo{Al fuoco di bivacco}
%autore{}
%album{}
%tonalita{Do}
%famiglia{Scout}
%gruppo{}
%momenti{}
%identificatore{al_fuoco_di_bivacco}
%data_revisione{2018_09_25}
%trascrittore{Federico Crippa}
%video{}
\beginsong{Al fuoco di bivacco}[by={}]
\beginverse 
\[C]Al fuoco di bivacco tutti \[A-]quanti siam raccolti,
la \[F]luce della fiamma rischiara i nostri \[C]volti.
Da quella fiamma pura scatu\[E7]risce una scin\[A-]tilla,
è la \[D-]fede \[G]di B.\[C]P. 
\endverse 

\beginchorus
Sempre avanti es\[A-]ploratori,
\[F]sempre in alto i nostri \[C]cuori.
Tutti gli scout \[E7]sono fra\[A-]telli
e can\[D-]tiamo sem\[G]pre co\[C]sì:
Ja\[C]mboree!
\endchorus
\beginverse 
Dal ^cielo ci sorridono e ci ^guardano le stelle,
son ^gaie e scintillanti, ci fan da senti^nelle.
Ma una che fra tutte si di^stingue è la più ^bella,
è la ^fede ^di B.^P. 
\endverse 

\beginverse 
L'es^ploratore è forte sia nel ^cuor che nella mente,
la ^sua parola è sacra, non sta senza far ^niente.
Egli è di buon umore e si ^arrangia a far di ^tutto:
è la ^legge ^di B.^P.
\endverse 
\endsong 

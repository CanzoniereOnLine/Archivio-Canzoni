%titolo{Teorema}
%autore{Marco Ferradini}
%album{}
%tonalita{LA-}
%famiglia{altre}
%gruppo{}
%momenti{}
%identificatore{teorema}
%data_revisione{}
%trascrittore{Francesco Raccanello}
\beginsong{Teorema}[by={Ferradini}]

\beginverse
\[A-]Prendi una donna \[F7+]dille che l'ami, 
\[G]scrivile canzoni d'a\[A-]more
\[A-]mandale rose \[F7+]e poesie, 
dalle \[G]anche spremute di \[E]cuore
falla \[A-]sempre sen\[F]tire impor\[G]tante, 
dalle il \[A-]meglio del \[F]meglio che \[G]hai
cerca \[A-]d'essere un \[F]tenero a\[G]mante,
 sii \[E7]sempre pre\[A-]sente, ri\[E-7]solvile i \[A-]guai.
E sta si\[A-]curo che ti \[F]lasce\[G]rà, 
chi è troppo a\[A-]mato \[F]amore non \[G]dà
e sta si\[A-]curo che ti \[F]lasce\[G]rà, 
chi meno \[A-]ama è il più \[E-7]forte, si \[A-]sa.
\endverse

\beginverse
^Prendi una donna ^trattala male, 
^lascia che ti aspetti per ^ore
^non farti vivo e ^quando la chiami, 
^fallo come fosse un fa^vore
fa sen^tire che è ^poco impor^tante, 
dosa ^bene amore ^e crudel^tà
cerca ^d'essere un ^tenero a^mante, 
ma ^fuori dal ^letto nes^suna pie^tà.
E allora ^sì vedrai che ^t'ame^rà, 
chi è meno a^mato più a^more ti ^dà
e allora ^sì vedrai che ^t'ame^rà, 
chi meno ^ama è il più ^forte si ^sa.
\endverse

\beginverse
^No caro amico non ^sono d'accordo, 
^parli da uomo fe^rito
^pezzo di pane ^lei se n'è andata 
^e tu non hai resi^stito
non e^sistono ^leggi in a^more,
 basta ^essere ^quello che ^sei
lascia a^perta la ^porta del ^cuore, 
ve^drai che una ^donna è già in ^cerca di ^te.
Senza l'a^more un uomo ^che co^s'è, 
su questo ^sarai d'ac^cordo con ^me
senza l'a^more un uomo ^che co^s'è, 
è questa ^l'unica ^legge che ^c'è.
\endverse

\endsong

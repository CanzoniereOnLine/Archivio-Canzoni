%titolo{Sono solo canzonette}
%autore{Edoardo Bennato}
%album{}
%tonalita{SOL}
%famiglia{Altre}
%gruppo{}
%momenti{}
%identificatore{sono_solo_canzonette}
%data_revisione{2020_03_25}
%trascrittore{Francesca Alampi}
\beginsong{Sono solo canzonette}[by={Bennato}]
\beginverse
\[G]Mi ricordo che anni fa 
\[B-]di sfuggita dentro un bar 
\[E7]ho sentito un juke-box che suo\[A-]na\[D7]va 
\[A-]e nei sogni \[D7]di bambino 
\[G]la chitarra era una \[E-]spada 
\[A]e chi non ci credeva era un \[A-]pira\[D7]ta! 
\[G]E la voglia di cantare 
\[B-]e la voglia di volare 
\[E7]forse mi è venuta proprio \[A-]allo\[D7]ra. 
\[A-]Forse è stata \[D7]una pazzia 
\[B-]però è l'unica \[E7]maniera 
di \[A-]dire sempre \[D7]quello che mi \[G]va. 
\endverse
\beginverse 
\[G]Non potrò mai diventare 
\[B-]direttore generale 
\[E7]delle poste o delle ferro\[A-]vie. \[D7] 
\[A-]Non potrò mai \[D7]far carriera 
\[G]nel giornale della\[E-] sera 
\[A]anche perché finirei in ga\[A-]le\[D7]ra! 
\[G]Mai nessuno mi darà 
\[B-]il suo voto per parlare 
\[E7]o per decidere del suo fu\[A-]tu\[D7]ro. 
\[A-]Nella mia ca\[D7]tegoria 
\[B-]è tutta gente \[E7]poco seria 
\[A-]di cui non ci \[D7]si può fi\[G]dare. 
\endverse 
\beginchorus
\[G] \[E-] \[A-] \[D7]
\endchorus
\beginverse
\[G]Guarda invece che scienziati, 
\[B-]che dottori, che avvocati, 
\[E7]che folla di ministri e depu\[A-]tati\[D7]! 
\[A-]Pensa che in que\[D7]sto momento 
\[G]proprio mentre io \[E-]sto cantando 
\[A]stanno seriamente lavo\[A-]ran\[D7]do! 
\[G]Per i dubbi e le domande 
\[B-]che ti assillano la mente 
\[E7]va' da loro e non ti preoccu\[A-]pare,\[D7] 
\[A-]sono a tua di\[D7]sposizione 
\[B-]e sempre, senza e\[E7]sitazione 
\[A-]loro ti \[D7]risponderanno. 
\endverse
\beginchorus 
\[G] \[E-] \[A-] \[D7]
\endchorus
\beginverse
\[F]Io di ri\[C]sposte non ne \[G]ho 
\[E]io faccio solo rock'n' \[A-]roll 
se ti con\[D]viene \[A7]bene,
\[G]io più di \[E-]tanto non posso \[A-]fa\[D7]re. 
\[G]Gli impresari di partito 
\[B-]mi hanno fatto un altro invito 
\[E7]e hanno detto che finisce \[A-]ma\[D7]le 
\[A-]se non vado \[D7]pure io 
\[G]al raduno \[E-]generale 
\[A-]della grande \[D7]festa nazio\[G]nale! 
\[G]Hanno detto che non posso 
\[B-]rifiutarmi proprio adesso 
\[E7]che anche a loro devo il mio suc\[A-]ces\[D7]so, 
\[A-]che son pazzo ed\[D7] incosciente 
\[B-]sono un irri\[E7]conoscente 
\[A-]un sovversivo, un \[D7]mezzo crimi\[G]nale. 
\endverse
\beginverse 
\[G]Ma che ci volete fare 
\[B-]non vi sembrerò normale, 
\[E7]ma è l'istinto che mi fa vo\[A-]la\[D7]re. 
\[A-]Non c'è gioco \[D7]né finzione 
\[G]perché l'uni\[E-]ca illusione 
è \[A7]quella della real\[E-]tà, della ragio\[A-]ne. \[D7] 
\[G]Però a quelli in malafede 
\[B-]sempre a caccia delle streghe 
\[E7]dico: no! non è una cosa \[A-]seria. \[D7] 
\[A-]E così è \[D7]se vi pare 
\[G]ma lasciatemi \[E]sfogare 
\[A-]non mettetemi \[D7]alle strette 
\[B-]e con quanto \[E]fiato ho in gola 
\[A-]vi urlerò: non \[D7]c'è paura! 
Ma \[G]che politica,\[G7] che cultura, 
\[A-]sono solo can\[D7]zonette
\[B-]non mettetemi \[E7]alle strette 
\[A-]sono s\dots{} sono s\dots{}
\[A-]sono solo \[D]canzo\[D7]nette.
\endverse
\endsong


%titolo{La locomotiva}
%autore{Francesco Guccini}
%album{}
%tonalita{RE}
%famiglia{altre}
%gruppo{}
%momenti{}
%identificatore{la_locomotiva}
%data_revisione{}
%trascrittore{Francesco Raccanello}
\beginsong{La locomotiva}[
	by={Guccini}]

\beginverse
Non \[D]so che viso avesse, \[A7]neppure come si chia\[D]mava, \[A]
con \[D]che voce parlasse, con \[A7]quale voce poi can\[D]tava, \[A7]
quanti \[G]anni avesse \[A]visto al\[D]lora, di \[G]che colore i \[A]suoi ca\[D]pelli,
ma \[G]nella fanta\[A]sia ho l'im\[F#-]magine \[B-]sua:
gli \[G]eroi son tutti \[A]giovani e \[D]belli,
gli \[G]eroi son tutti \[A]giovani e \[D]belli,
gli \[G]eroi son tutti \[A]giovani e \[D]belli
\endverse

\chordsoff
\beginverse
Conosco invece l'epoca dei fatti, qual' era il suo mestiere:
i primi anni del secolo, macchinista, ferroviere,
i tempi in cui si cominciava la guerra santa dei pezzenti
sembrava il treno anch' esso un mito di progresso
lanciato sopra i continenti,
lanciato sopra i continenti,
lanciato sopra i continenti...
\endverse


\beginverse
E la locomotiva sembrava fosse un mostro strano
che l'uomo dominava con il pensiero e con la mano:
ruggendo si lasciava indietro distanze che sembravano infinite,
sembrava avesse dentro un potere tremendo,
la stessa forza della dinamite,
la stessa forza della dinamite,
la stessa forza della dinamite..
\endverse

\beginverse
Ma un' altra grande forza spiegava allora le sue ali,
parole che dicevano "gli uomini son tutti uguali"
e contro ai re e ai tiranni scoppiava nella via
la bomba proletaria e illuminava l' aria
la fiaccola dell' anarchia,
la fiaccola dell' anarchia,
la fiaccola dell' anarchia...
\endverse

\beginverse
Un treno tutti i giorni passava per la sua stazione,
un treno di lusso, lontana destinazione:
vedeva gente riverita, pensava a quei velluti, agli ori,
pensava al magro giorno della sua gente attorno,
pensava un treno pieno di signori,
pensava un treno pieno di signori,
pensava un treno pieno di signori...
\endverse

\beginverse
Non so che cosa accadde, perché prese la decisione,
forse una rabbia antica, generazioni senza nome
che urlarono vendetta, gli accecarono il cuore:
dimenticò pietà, scordò la sua bontà,
la bomba sua la macchina a vapore,
la bomba sua la macchina a vapore,
la bomba sua la macchina a vapore...
\endverse

\chordson
\transpose{2}
\beginverse
E ^sul binario st^ava la locomo^tiva, ^
la ^macchina pulsante ^sembrava fosse cosa ^viva, ^
sem^brava un giovane pu^ledro che ap^pena libe^rato il ^freno
mor^desse la ro^taia con ^muscoli d' ac^ciaio,
con ^forza cieca di ba^leno,
con ^forza cieca di ba^leno,
con ^forza cieca di ba^leno...
\endverse
\chordsoff

\beginverse
E un giorno come gli altri, ma forse con più rabbia in corpo
pensò che aveva il modo di riparare a qualche torto.
Salì sul mostro che dormiva, cercò di mandar via la sua paura
e prima di pensare a quel che stava a fare,
il mostro divorava la pianura,
il mostro divorava la pianura,
il mostro divorava la pianura...
\endverse

\beginverse
Correva l' altro treno ignaro e quasi senza fretta,
nessuno immaginava di andare verso la vendetta,
ma alla stazione di Bologna arrivò la notizia in un baleno:
"notizia di emergenza, agite con urgenza,
un pazzo si è lanciato contro al treno,
un pazzo si è lanciato contro al treno,
un pazzo si è lanciato contro al treno..."
\endverse

\chordson
\transpose{1}
\beginverse
Ma ^intanto corre, ^corre, corre la locomo^tiva ^
e ^sibila il vapore e ^sembra quasi cosa ^viva ^
e ^sembra dire ai ^contadini ^curvi 
il ^fischio che si ^spande in ^aria:
"Fra^tello, non te^mere, che ^corro al mio do^vere!
^Trionfi la giu^stizia prole^taria!
^Trionfi la gius^tizia prole^taria!
^Trionfi la gius^tizia prole^taria!"
\endverse

\transpose{-1}
\beginverse
E in^tanto corre corre ^corre sem^pre più ^forte
e ^corre corre ^corre corre verso la ^morte ^
e ^niente ormai può ^tratte^nere l' im^mensa forza ^distrut^trice,
as^petta sol lo ^schianto e ^poi che giunga il ^manto
della ^gran^de consola^trice,
della ^gran^de consola^trice,
della ^gran^de consola^trice...
\endverse

\transpose{-2}
\beginverse
La ^storia ci racconta ^come finì la ^corsa ^
la ^macchina deviata ^lungo una linea ^morta ^
con l' ^ultimo suo ^grido d' ani^male 
la ^macchina eruttò ^lapilli e ^lava,
es^plose contro il ^cielo, poi il ^fumo sparse il ^velo:
lo ^raccolsero che ^ancora respi^rava,
lo rac^colsero che ^ancora respi^rava,
lo rac^colsero che ^ancora respi^rava...
\endverse
\chordsoff

\beginverse
Ma a noi piace pensarlo ancora dietro al motore
mentre fa correr via la macchina a vapore
e che ci giunga un giorno ancora la notizia
di una locomotiva, come una cosa viva,
lanciata a bomba contro l' ingiustizia,
lanciata a bomba contro l' ingiustizia,
lanciata a bomba contro l' ingiustizia
\endverse

\endsong

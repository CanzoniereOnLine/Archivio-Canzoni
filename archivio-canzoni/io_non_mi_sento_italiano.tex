%titolo{Io non mi sento italiano}
%autore{Giorgio Gaber}
%album{}
%tonalita{LA-}
%famiglia{Altre}
%gruppo{}
%momenti{}
%identificatore{io_non_mi_sento_italiano}
%data_revisione{2020_03_27}
%trascrittore{Francesca Alampi}
\beginsong{Io non mi sento italiano}[by={Gaber}]
\capo{5}
\beginverse*
\[A-]Io, G.G., sono nato e vivo a Milano \chordsoff
io non mi sento italiano, \brk ma per fortuna o purtroppo lo sono
\endverse 
\beginverse 
Mi \[A-]scusi presidente non è per colpa \[G]mia
ma questa nostra \[F]patria non so che cosa \[E7]sia
può darsi che mi \[A-]sbagli che sia una bella i\[G]dea
ma temo che di\[F]venti una brutta poe\[E7]sia.
Mi scusi presi\[C]dente non sento un gran bi\[G]sogno
dell'inno nazio\[A-]nale di cui un po' mi ver\[E7]gogno
in quanto i calcia\[F]tori non voglio giudi\[C]care
i nostri non lo \[D-]sanno o hanno più pu\[E7]dore-
\endverse 
\beginchorus
\[A-]Io non mi sento italiano, \brk ma per for\[D-]tuna o pur\[E7]troppo lo \[A-]son.
\endchorus
\beginverse 
Mi \[A-]scusi presidente se arrivo all'impu\[G]denza
di dire che non \[F]sento alcuna apparte\[E7]nenza
e tranne Gari\[A-]baldi e altri eroi glo\[G]riosi
non vedo alcun mo\[F]tivo per essere orgo\[E7]gliosi.
Mi scusi presi\[C]dente, ma ho in mente il fana\[G]tismo
delle camicie \[A-]nere al tempo del fa\[E7]scismo
e da cui un bel giorno \[F]nacque questa democra\[C]zia
che a farle i compli\[D-]menti ci vuole fanta\[E7]sia .
\endverse 
\beginchorus
\[A-]Io non mi sento italiano, \brk ma per for\[D-]tuna o pur\[E7]troppo lo \[A-]son.
\[C]Questo bel pa\[G]ese \[A-]pieno di poe\[E7]sia \[F]ha tante pre\[C]tese
\[D-]ma nel nostro \[E7]mondo occidentale è la periferia.
\endchorus 
\beginverse
Mi \[A-]scusi presidente ma questo nostro \[G]stato
che voi rappresen\[F]tate mi sembra un po' sfa\[E7]sciato
è anche troppo \[A-]chiaro agli occhi della \[G]gente
che è tutto calco\[F]lato e non funziona \[E7]niente
sarà che gli ita\[C]liani per lunga tradi\[G]zione
son troppo appassio\[A-]nati di ogni discus\[E7]sione
persino in parla\[F]mento c'è un'aria incan\[C]descente
si scannano su \[D-]tutto e poi non cambia \[E7]niente.
\endverse 
\beginchorus
\[A-]Io non mi sento italiano, \brk ma per for\[D-]tuna o pur\[E7]troppo lo \[A-]son.
\endchorus 
\beginverse
Mi \[A-]scusi presidente dovete conve\[G]nire
che i limiti che ab\[F]biamo ce li dobbiamo \[E7]dire
a parte il disfat\[A-]tismo noi siamo quel che \[G]siamo
e abbiamo anche un pas\[F]sato che non dimenti\[E7]chiamo.
Mi scusi presi\[C]dente ma forse noi ita\[G]liani
per gli altri siamo \[A-]solo spaghetti e mando\[E7]lini
allora qui m'in\[F]cazzo, son fiero e me ne \[C]vanto
e gli sbatto sulla \[D-]faccia cos'è il rinasci\[E7]mento.
\endverse 
\beginchorus
\[A-]Io non mi sento italiano, \brk ma per for\[D-]tuna o pur\[E7]troppo lo \[A-]son.
\[C]Questo bel pa\[G]ese \[A-]forse è poco \[E7]saggio \[F]ma ha le idee con\[C]fuse
\[D-]ma se fossi \[E7]nato in altri luoghi poteva andarmi peggio.
\endchorus 
\beginverse 
Mi \[A-]scusi presidente ormai ne ho dette \[G]tante
c'è un'altra osserva\[F]zione che credo sia impor\[E7]tante
rispetto agli stra\[A-]nieri noi ci crediamo \[G]meno
ma forse abbiam ca\[F]pito che il mondo è un \[E7]teatrino.
Mi scusi presi\[C]dente lo so che non gio\[G]ite
se il grido Italia I\[A-]talia c'è solo alle par\[E7]tite
ma un po' per non mo\[F]rire o forse un po' per \[C]celia
abbiam fatto l'Eu\[D-]ropa facciamo anche l'\[E7]Italia.
\endverse
\beginchorus
\[A-]Io non mi sento italiano, \brk ma per for\[D-]tuna o pur\[E7]troppo lo \[A-]son.
\[A-]Io non mi sento italiano, \brk ma per for\[D-]tuna o pur\[E7]troppo 
per for\[D-]tuna o pur\[E7]troppo per fortuna, per fortuna lo \[A-]son.
\endchorus 
\endsong

%titolo{Niente di speciale}
%autore{Lo stato sociale}
%album{Amore, lavoro e altri miti da sfatare}
%tonalita{Do}
%famiglia{Altre}
%gruppo{}
%momenti{}
%identificatore{niente_di_speciale}
%data_revisione{2018_02_04}
%trascrittore{Chiara Galafassi}
%video{https://www.youtube.com/watch?v=fy6Fgyi9fHw}
\beginsong{Niente di speciale}[by={Lo stato sociale}]
\beginverse
Come \[C]faccio a dirti che non mi piace
il nome di tua so\[F]rella, il tuo freno a motore
il tuo tenermi na\[C]scosto agli occhi del mondo
quando è il \[G]mondo che non sai guardare?
E tutti i tuoi con\[A-]sigli servono a poco,
sono bra\[F]vissimo a sbagliare da solo.
Come faccio a tenere lon\[C]tana questa canzone da chi
\[G]non la deve ascoltare?
Se sa\[C]pesse quanto ho scritto di te
ti farebbe un con\[F]tratto il mio editore.
Mi porteresti a Sara\[C]jevo
nell'autunno dei 30 anni
e non do\[G]vresti più lavorare.
E cammino al te\[A-]lefono su un giro di Do
anche adesso che un \[F]po' ho imparato a suonare
perché sei come \[C]me
più sei leggera
meno \[G]sei superficiale.
\endverse
\beginchorus
Ti ho so\[A-]gnato in un ufficio FS,
cantavi in fran\[F]cese allo sportello reclami
ti prendevano in \[C]giro tutti i miei amici \brk travestiti \[G] da ballerine e da nani.
Di che \[A-]cosa hai paura davvero?
Forse che la \[F]gatta scappi per le scale?
Non sai quanto in\[C]vidio gli animali
loro capiscono \[G]sempre da chi tornare.
Vorrei una do\[A-]menica pomeriggio
per ogni lune\[F]dì che non ho saputo iniziare,
ma siamo una \[C]storia che non si può dire
non abbiamo \[G]niente di speciale.
Non fosse che \[A-]io ho paura di crescere
e tu \[F]quella di nuotare
e sai \[C]dirmi che mi ami, ma solo finché
non si \[G]esce dall'ascensore.
Eppure lo \[A-]senti anche tu
che abbiamo \[G]fatto
lo stesso \[C]errore.
\endchorus
\beginverse
Lo ^sai che chi ci dorme nei letti
ha la bocca a^perta per abboccare?
Sai che è ^facile odiare il terremoto
il dif^ficile è ricostruire?
Sai che ho pro^vato pena per te
non scegliere, ^scegliere di subire, 
non è so^gnare che aiuta a vivere
è vivere che ^deve aiutarti a sognare.
E allora ^tieniti pure la coperta
sono bravissimo ad a^vere freddo da solo,
tieniti il tuo ego^ismo discreto 
se non sei capace di a^verlo alla luce del sole.
Tieniti le mie pa^role
che hai 35 metri ^quadri da arredare.
Anzi ^tienimi ancora i capelli, senza te ^non so più
respirare.
\endverse

\beginchorus
Ti ho so\[A-]gnato in un ufficio FS,
cantavi in fran\[F]cese allo sportello reclami
ti prendevano in \[C]giro tutti i miei amici \brk travestiti \[G] da ballerine e da nani.
Di che \[A-]cosa hai paura davvero?
Forse che la \[F]gatta scappi per le scale?
Non sai quanto in\[C]vidio gli animali
loro capiscono \[G]sempre da chi tornare.
Vorrei una do\[A-]menica pomeriggio
per ogni lune\[F]dì che non ho saputo iniziare,
ma siamo una \[C]storia che non si può dire
non abbiamo \[G]niente di speciale.
Non fosse che \[A-]io ho paura di crescere
e tu \[F]quella di nuotare
e sai \[C]dirmi che mi ami, ma solo finché
non si \[G]esce dall'ascensore.
Eppure lo \[A-]senti anche tu
che abbiamo \[G]fatto
lo stesso \[C]errore.
\endchorus

\ifchorded
\beginverse*
\vspace*{-\versesep}
{\nolyrics \[F] \[F]\[C] \[G] \rep{2}}
\endverse
\fi

\beginverse
\[F]Tienimi le mani
non an\[C]neghe\[G]rai. \rep{5}
\[F]Potrà capitarti di bere
ma \[C]non anneghe\[G]rai.
\[F]Ogni volta che scegli, tu scegli
il tipo di \[C]schiavo che non sa\[G]rai
\endverse

\beginverse*
\[F]\[F] \[C] \[G] 
\endverse
\endsong

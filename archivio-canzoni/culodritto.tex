%titolo{Culodritto}
%autore{Francesco Guccini}
%album{Signora Bovary}
%tonalita{DO}
%famiglia{Altre}
%gruppo{}
%momenti{}
%identificatore{culodritto}
%data_revisione{2023_05_30}
%trascrittore{Francesco Endrici}
%video{https://www.youtube.com/watch?v=K2zDugsPO9Q}
\beginsong{Culodritto}[by={Guccini}]
\beginverse
Ma come vorrei avere i tuoi \[C]occhi,
\[F]spalancati sul mondo come carte assorbenti
\[G]e le tue risate pulite e piene,
quasi senza rimorsi o penti\[F]menti,
ma \[C]come vorrei avere da guardare
ancora \[G]tutto come i libri da sfogliare
e avere ancora \[F]tutto, o quasi tutto, da pro\[C]vare\dots{} \[G7]
\endverse

\beginverse
\[C]Culodritto, che vai via sicura,
\[F]trasformando dal vivo cromosomi corsari
di longo\[G]bardi, di celti e romani
dell'antica pianura, di \[F]montanari,
regi\[C]netta dei telecomandi,
di gnosi asso\[G]lute che asserisci e domandi,
di so\[F]spetto e di fede nel \[D-7]mondo curioso dei \[G]grandi,
\endverse

\beginverse    
anche \[G7]se non a\[C]vrai
le mie risse terrose di campi, cortile e di \[G]strade \[E7]
e \[A-]non sa\[E-]prai
che sa\[A-]pore ha il sapore dell'\[D7]uva rubata a un fi\[G]lare,
presto \[G]ti accorge\[C]rai
com'è facile farsi un'inutile software di \[G]scienza \[E7]
\[A-]e ve\[E-]drai
che con\[A-]fuso problema è ado\[D7]prare la propria espe\[G]rienza\dots{}
\[F]Culodritto, cosa vuoi che ti dica?
Solo che \[C]costa sempre fatica
e che il \[D7]vivere è sempre quello,
ma è storia an\[G]tica, Culo\[C]dritto\dots{}
\endverse
\transpose{1}\preferflats
\beginverse
dammi ancora la \[C]mano,
\[F]anche se quello stringerla è solo un pretesto
per sen\[G]tire quella tua fiducia totale
che nessuno mi ha dato \[F]o mi ha mai chiesto;
vola, vola \[C]tu, dov'io vorrei volare
verso un \[G]mondo dove è ancora tutto da fare
e dove è ancora \[F]tutto, o quasi tutto\dots{}
vola, vola \[C]tu, dov'io vorrei volare
verso un \[G]mondo dove è ancora tutto da fare
e dove è ancora \[F]tutto, o quasi \[F#/G]tutto, da sba\[C]gliare\dots{}
\endverse
\endsong

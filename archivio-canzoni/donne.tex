%titolo{Donne}
%autore{}
%album{}
%tonalita{DO}
%famiglia{altre}
%gruppo{}
%momenti{}
%identificatore{donne}
%data_revisione{}
%trascrittore{Francesco Raccanello}
\beginsong{Donne}[by = {Zucchero}]

\beginverse
\[C]Donne du du \[G]du, in \[A-]cerca di \[F]guai
\[C]Donne a un te\[G]lefono che \[A-]non suona \[G]mai
\[C]Donne du du \[G]du, in \[A-]mezzo a una \[F]via
\[C]donne allo \[G]sbando senza \[A-]compa\[G]gnia
Negli occhi \[C]hanno dei con\[G]sigli 
e tanta \[A-]voglia di avven\[C]ture
e se hanno \[F]fatto molti \[C]sbagli sono \[G]piene di paure.
Le vedi \[C]camminare in\[G]sieme 
nella \[A-]pioggia o sotto il \[C]sole
dentro \[F]pomeriggi o\[C]pachi 
senza \[C]gioie nè dolore
\endverse

\beginverse
^Donne du du \[G]du, pia^neti dis^persi
^per tutti gli ^uomini co^sì di^versi
^Donne du du \[G]du, a^miche di ^sempre
^donne alla ^moda donne ^contro cor^rente
Negli occhi ^hanno gli aero^plani
per vo^lare ad alta ^quota
dove ^si respira l'^aria e la ^vita non è vuota
Le vedi ^camminare in^sieme
nella ^pioggia o sotto il ^sole
dentro ^pomeriggi o^pachi
senza ^gioie nè dolore
\[F]Donne \[G] \[A-] \[F]Donne  \[G]
\endverse

\beginverse
\[C]Donne du du \[G]du, in \[A-]cerca di \[F]guai
\[C]Donne a un te\[G]lefono che \[A-]non suona \[G]mai
\[C]Donne du du \[G]du, in \[A-]mezzo a una \[F]via
\[C]donne allo \[G]sbando senza \[A-]compa\[G]gnia
\[C]Donne du du \[G]du \[A-] \[F]
\endverse

\endsong

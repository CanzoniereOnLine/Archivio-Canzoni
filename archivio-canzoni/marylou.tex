%titolo{Marylou}
%autore{Mannarino}
%album{}
%tonalita{DO}
%famiglia{altre}
%gruppo{}
%momenti{}
%identificatore{marylou}
%data_revisione{}
%trascrittore{Francesco Raccanello}
\beginsong{Marylou}[
	by={Mannarino}]
	
\beginchorus
La donna del porto balla con l'abito \[C]corto,
\[D-]rossa nella \[F]sera \[G] se ne \[A-]va.
\endchorus

\beginverse
\memorize
U\[E]na mattina ha preso la co\[A-]rriera
per\[E]ché voleva andare alla \[A-]città,
che il \[F]padre con la zappa le ru\[C]bava tutta \[E]la feli\[A-]cità.
Trova\[E]ta fu rinchiusa in un con\[A-]vento,
\[E]però poi fuggì dall'aldi\[A-]là
e in qu\[F]esta strada sporca come il \[C]mondo quanto è \[E]bello cammi\[A-]nar.
\endverse

\beginchorus
La \[A-]donna del porto balla con l'abito \[C]corto,
\[D-]rossa nella \[F]sera \[G] se ne \[A-]va.
\endchorus

\beginverse
Il gi^orno vende al sole del mer^cato
il ^sale dolce della liber^tà,
la ^notte vola sopra un cane ^alato e sveglia ^tutta la ^città.
^È vestita sempre tras^parente
^e i capelli sono di ^lillà,
^porta in mano una stella ca^dente e terro^rizza la ^città.
\endverse

\beginchorus
\[A-]Ma\[G]ry\[C]lou,  \[C]Ma\[G]ry\[A-]lou,  
\[A-]tutti \[G]i mari\[D-]nai    \[F]gridano "\[G]I love \[A-]you!" \rep{2}
\endchorus

\beginverse
\nolyrics Stacco: \[A-] \[G] \[C] \[C] \[G] \[A-] \[A-] \[G] \[D-] \[F] \[G] \[A-]
\endverse

\beginchorus
La donna del porto balla con l'abito \[C]corto,
\[D-]rossa nella \[F]sera \[G] se ne \[A-]va.
\endchorus

\beginverse
Ma poi un ragioniere ha svalvo\[A-]lato
pe\[E]rché non riusciva più a con\[A-]tar,
\[F]poi precisamente ha calco\[C]lato di amma\[E]zzarla dentro a un \[A-]bar.
\[E]Era stesa sopra al pavi\[A-]mento
ma si \[E]è rialzato e ha detto: "Non si \[A-]fa!".
Gli ha \[F]dato un pizzicotto sotto il \[C]mento e lo ha spe\[E]dito all'aldi\[A-]là.
\endverse


\beginchorus
La \[A-]donna del porto balla con l'abito \[C]corto,
\[D-]rossa nella \[F]sera \[G] se ne \[A-]va.
\endchorus

\beginverse
A ^volte al gregge infame della ^gente
^serve un lupo nero da ammae^strar,
il ^pazzo a fatto tutto di sua ^sponte ma aveva gli ^occhi di mam^mà.
Pur^troppo Marylou non l'ho più ^vista
e a ^volte io mi chiedo dove ^sta,
ma ^forse è meglio vivere all'in^ferno che in una san^tissima ^città.
\endverse

\beginchorus
\[A-]Ma\[G]ry\[C]lou,  \[C]Ma\[G]ry\[A-]lou,  
\[A-]tutti \[G]i mari\[D-]nai    \[F]gridano "\[G]I love \[A-]you!" \rep{2}
\endchorus

\endsong

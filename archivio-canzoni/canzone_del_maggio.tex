%titolo{Canzone del maggio}
%autore{Fabrizio De André}
%album{}
%tonalita{LA-}
%famiglia{Altre}
%gruppo{}
%momenti{}
%identificatore{canzone_del_maggio}
%data_revisione{2020_03_10}
%trascrittore{Francesca Alampi}
\beginsong{Canzone del maggio}[by={De André}]
\beginverse
\[G]Anche se il nostro maggio  \brk  ha fatto a \[C]meno del \[D]vostro co\[G]raggio,
se la paura di guar\[C]dare  \brk vi ha \[D]fatto chinare il \[G]mento,
se il \[C]fuoco ha risparmi\[G]ato  \brk le \[D]vostre Mille\[G]cento,
anche se \[C]voi vi credete as\[G]solti
siete lo \[D]stesso coinvol\[G]ti.  \[C]   \[D]   \[G]
\endverse
\beginverse
^E se vi siete detti  \brk non ^sta suc^cedendo ^niente
le fabbriche riapri^ranno  \brk arreste^ranno qualche stu^dente
con^vinti che fosse un ^gioco \brk  a cui a^vremmo gioca^to poco
provate ^pure a credervi a^solti
siete lo ^stesso coin^volti.  ^   ^   ^
\endverse
\beginverse
^Anche se avete chiuso le vostre ^porte  \brk sul ^nostro ^muso
la notte ^che le pantere  \brk ci mor^devano il se^dere
la^sciamoci in buona^fede \brk  massacrare ^sui marcia^piedi
anche se ^ora ve ne fre^gate
voi quella ^notte voi c'era^vate. ^   ^   ^
\endverse
\beginverse
^E se nei vostri quartieri \brk  tutto è ri^masto ^come ^ieri,
senza le barri^cate, senza fe^riti, \brk  senza gra^nate,
se a^vete preso per ^buone \brk  le veri^tà della tele^visione
anche se al^lora vi siete as^solti
siete lo ^stesso coin^volti.^   ^   ^
\endverse
\beginverse
\[G]E se credete ora,  \brk che \[C]tutto \[D]sia come \[G]prima
perché avete votato an\[C]cora  \brk la sicu\[D]rezza, la disci\[G]plina,
con\[C]vinti di allon\[G]tanare \brk  la pa\[D]ura di cam\[G]biare
verremo an\[C]cora alle \[G]vostre porte \brk e gride\[D4]remo an\[D]cora più \[G]forte
per quanto \[C]voi vi crediate as\[G]solti  \brk siete per \[D4]sempre coinvo\[G]lti,
per quanto \[C]voi vi crediate as\[G]solti
siete per \[D]sempre coin\[G]volti. \[C]   \[D]   \[G]
\endverse
\endsong
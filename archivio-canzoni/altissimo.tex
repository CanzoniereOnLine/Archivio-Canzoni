%titolo{Altissimo}
%autore{Spoladore}
%album{Unanima}
%tonalita{Re}
%famiglia{Liturgica}
%gruppo{}
%momenti{Comunione;San Francesco}
%identificatore{altissimo}
%data_revisione{2011_12_31}
%trascrittore{Francesco Endrici - Manuel Toniato}
\beginsong{Altissimo}[by={Spoladore}]
\ifchorded
\beginverse*
\vspace*{-0.8\versesep}
{\nolyrics \[D4/9] \[D] \[A] \[D4/9]}
\vspace*{-\versesep}
\endverse
\fi

\beginverse
\[D9]Altissimo, Onni\[B-4/7]potente, Buon Si\[D9]gnore, \[B-4/7] 
tue son le \[D9]lodi, la gloria, l'o\[B-4/7]nore \brk e ogni benedi\[D9]zione, \[B-4/7] 
che a Te \[D9]solo e al tuo Nome Al\[B-4/7]tissimo \brk possiamo ele\[D9]vare,\[B-4/7]  
e nessun \[D9]uomo può credersi \[B-4/7]degno \brk di poterti no\[F#-4/7]minare.   \[E-7] 
\endverse

\beginverse
\chordsoff
Laudato sii, mi Signore con tutte le tue creature,
specialmente per frate sole così bello e radioso,
con la sua luce illumini il giorno ed illumini noi
e con grande splendore ci parla di Te Signore.
\endverse

\beginchorus
Lo\[D]date, 
bene\[A]dite il \[B-7]Signore, ringra\[G]ziate e ser\[D]vite 
con \[A]grande umil\[B-7]tà. 
Lo\[G]date, lo\[D]date, 
bene\[A]dite il Si\[B-7]gnore, con \[G]grande umil\[D]tà,
ringra\[A]ziate e ser\[B-4/7]vite, \brk con \[G]grande umil\[F#-4/7]{tà.} \[E-7] 
\endchorus

\beginverse
\chordsoff
Laudato sii, mi Signore, per sora luna e le stelle,
le hai formate nel cielo così chiare preziose e belle.
Per frate vento, per l'aria e il sereno \brk ed ogni tempo.
Così la Vita Tu cresci e sostieni in ogni tua creatura.
\endverse

\beginverse
\chordsoff
Laudato sii, mi Signore, per sora acqua così preziosa,
per frate fuoco giocoso e potente \brk che ci illumina la notte.
Laudato sii, mi Signore, per sora nostra \brk madre la terra,
ci sostiene, governa e ci dona fiori, frutti ed erba.
\endverse

\beginverse
\chordsoff
Laudato sii, mi Signore per quelli che \brk per il tuo Amore
perdonano e sopportano in pace \brk ogni persecuzione,
che sora morte ha trovato viventi \brk nella tua volontà,
da Te Altissimo un giorno saranno \brk da Te incoronati.
\endverse
\endsong


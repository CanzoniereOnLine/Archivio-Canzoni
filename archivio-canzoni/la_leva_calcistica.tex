%titolo{La leva calcistica della classe '68}
%autore{Francesco De Gregori}
%album{Titanic}
%tonalita{RE}
%famiglia{Altre}
%gruppo{}
%momenti{}
%identificatore{la_leva_calcistica}
%data_revisione{2020_03_10}
%trascrittore{Francesca Alampi}
\beginsong{La leva calcistica della classe '68}[by={De Gregori}]
\ifchorded
\beginverse*
\vspace*{-0.8\versesep}
{\nolyrics \[D] \[D7+] \[B-] \[D] \[G] \[A]}
\endverse
\fi
\beginverse
\[D] Sole sul tetto dei pa\[D7+]lazzi in costruzione\[B-] \brk sole che batte sul \[D]campo di pallone
\[G] e terra e polvere che \[G6]tira vento \[G]e poi magari pio\[A]ve.
\endverse
\beginverse 
\[D] Nino cammina che \[D7+]sembra un uomo,\[B-] \brk con le scarpette di \[D]gomma dura \[G] 
dodici anni e il cuore \[G6]pieno di paura.\[A] 
\endverse
\beginchorus
Ma \[G-]Nino non aver pa\[C]ura di sba\[F]gliare un calcio di rigore
non è \[A4]mica da questi partico\[A]lari \brk che si \[D-]giudica un giocatore
un gioca\[B&]tore lo vedi dal co\[C]raggio, \brk dall'altru\[B&]ismo, dalla fanta\[C]sia. \[C7] 
\endchorus
\beginverse
E chis\[F]sà quanti ne hai visti, \brk quanti ne vedrai di gioca\[A-]tori tristi
che non hanno vinto mai ed hanno ap\[A-7]peso le scarpe \brk a qualche tipo di muro
e adesso \[D4]ridono dentro al \[D]bar.
E \[B&]sono innamorati da dieci anni, \brk con una \[G-]donna che non hanno amato \[C]mai
chissà \[F]quanti ne hai veduti, \brk chissà \[A4]quanti ne vedrai. \[A] 
\endverse
\beginverse
\[D] Nino capì fin dal \[D7+]primo momento,\[B-] \brk l'allenatore sem\[D]brava contento
e allora \[G] mise il cuore dentro \[G6]alle scarpe \brk e \[G]corse più veloce del vento.\[A] 
\endverse
\beginverse
\[D] Prese un pallone che sem\[D7+]brava stregato,\[B-] \brk accanto al piede rima\[D]neva incollato
\[G] entrò nell'area, tirò \[G6]senza guardare \brk ed il por\[A4]tiere glielo fece passare. \[A] \[A4] \[A] 
\endverse
\beginchorus
\[G-]Nino non aver pa\[C]ura di ti\[F]rare un calcio di rigore
non è \[A4]mica da questi partico\[A]lari \brk che si \[D-]giudica un giocatore
un gioca\[B&]tore lo vedi dal co\[C]raggio,\brk  dall'altru\[B&]ismo, dalla fanta\[C]sia.\ [C7] 
\endchorus
\beginverse
\[B&]Il ragazzo si farà, anche se \[C]ha le spalle strette
quest'altro \[F]anno giocherà, con la \[A4]maglia numero \[A]sette.
\endverse
\endsong
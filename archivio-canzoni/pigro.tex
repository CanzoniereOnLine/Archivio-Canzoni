%titolo{Pigro}
%autore{Ivan Graziani}
%album{Pigro}
%tonalita{MI}
%famiglia{Altre}
%gruppo{}
%momenti{}
%identificatore{pigro}
%data_revisione{2020_03_10}
%trascrittore{Francesca Alampi}
\beginsong{Pigro}[by={Graziani}]
\beginverse
\[E]  Tu sai ci\[A]tare \[E]  i classici a me\[A]moria
\[E]  ma non di\[A]stingui il \[C]ramo \[D]da una \[A]foglia
il \[C]ramo \[D]da una \[A]foglia, pigro.  \[E]  \[A]  \[E]  \[A]  \[E]  \[A]  \[E]  \[A] 
\endverse
\beginverse
\[E]  Una mente \[A]fertile   \[E]dici “È alla \[A]base”,
\[E]  ma la tua \[A]scienza ha cre\[C]ato \[D]l'igno\[A]ranza
ha cre\[C]ato \[D]l'igno\[A]ranza, pigro. \[E]  \[A]  \[E]  \[A]  \[E]  \[A]  \[E]  \[A] 
\endverse
\beginchorus
E \[D]poi le parolacce che ti lasci scappare
che \[E]servono a condire il tuo discorso d'autore
come \[D]bava di lumache stanno lì a dimostrare
che è \[C#-]vero, è \[F#-]vero, non si può miglio\[C#-]rare
col tuo \[B-]schifo d'e\[E]duca\[A]zione,
col tuo \[B-]schifo d'e\[E]ducaz\[A]ione, pigro. \[E]   \[A]   \[E]   \[A]  \[E]  \[A]  \[E]  \[A] 
\endchorus
\beginverse
\[E]  La capra per il \[A]latte, \[E]  la donna per le \[A]voglie
\[E]  ma non ti ac\[A]corgi della \[C]noia che \[D]ha tua \[A]moglie
della \[C]noia che \[D]ha tua \[A]moglie.
\endverse
\beginverse 
\[E]  Tu castighi i \[A]figli \[E]  in maniera esem\[A]plare
\[E]  poi dici siamo \[A]liberi, nes\[C]suno deve \[D]giudi\[A]care
nessuno \[C]deve \[D]giudi\[A]care, pigro.  \[E]  \[A]  \[E]  \[A]  \[E]  \[A]  \[E]  \[A] 
\endverse
\beginchorus
E \[D]poi le parolacce che ti lasci scappare
che \[E]servono a condire il tuo discorso d'autore
come \[D]bava di lumache stanno là a dimostrare
che è \[C#-]vero, è \[F#-]vero, non si può miglio\[C#-]rare
col tuo \[B-]schifo d'e\[E]duca\[A]zione,
col tuo \[B-]schifo d'e\[E]duca\[A]zione, pigro.\[E]   \[A]   \[E]   \[A] 
\endchorus
\endsong
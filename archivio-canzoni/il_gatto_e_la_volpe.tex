%titolo{Il gatto e la volpe}
%autore{Edoardo Bennato}
%album{}
%tonalita{DO}
%famiglia{altre}
%gruppo{}
%momenti{}
%identificatore{il_gatto_e_la_volpe}
%data_revisione{}
%trascrittore{Francesco Raccanello}
\beginsong{Il gatto e la volpe}[by={Bennato}]

\beginverse
\[C]Quanta fretta, ma \[A-]dove \[D-7]corri, do\[G]ve vai 
\[C]{se }ci ascolti per \[A-]{un }momento, \[D-7]capi\[G]rai, 
\[C]{lui }e il gatto, ed \[E7]{io }la volpe, 
\[A-]stiamo in società di \[D-7]noi ti \[G7]puoi fidar \[A-] \[C] \[A-] 
\endverse

\beginverse
\[C]Puoi parlarci dei \[A-]tuoi problemi, \[D-7]dei tuoi \[G]guai 
\[C]{i }migliori in \[A-]questo campo, \[D-7]siamo \[G]noi 
\[C]{è }una ditta spe\[E7]cializzata, \[A-]{fa }un contratto e vedrai
che \[D-7]non ti \[G7]penti\[C]rai \[A-] \[C] \[A-]
\endverse

\beginverse
\[C]Noi scopriamo ta\[A-]lenti e non \[D-7]sbagliamo \[G7]sol 
\[C]noi sapremo sfrut\[A-]tare le \[D-7]tue quali\[G]tà 
\[C]dacci solo \[E7]quattro monete 
e \[A7]ti iscriviamo al concorso 
per \[D-7]la ce\[G7]lebri\[C]tà! \[A-] \[C] \[A7]
\endverse

\beginchorus
\[Fa]Non vedi che è un \[G7]vero affare 
\[C]non perdere \[A-]l'occasione 
\[D-7]se no poi \[G7]te ne pentirai\[C7] 
\[Fa]non capita tutti \[G7]i giorni
\[E-]di avere \[A-]due consulenti 
\[D7]due impresari, \[G7]che si fanno 
in quattro per te! 
\endchorus

\beginverse
^Avanti, non ^perder tempo, ^firma ^qua 
^è un normale con^tratto è una for^mali^tà 
^tu ci cedi ^tutti i diritti 
e ^noi faremo di ^te 
un ^divo da ^hit pa^rade!
\endverse

\beginverse
^Quanta fretta, ma ^dove corri; ^dove ^vai 
^che fortuna che hai a^vuto ad incon^trare ^noi 
^lui e il gatto, ed ^io la volpe, ^stiamo in socie^tà 
di ^noi ti ^puoi fi^dar!\dots di \[D-]noi ti \[G]puoi fi\[C]dar!
\endverse
\endsong

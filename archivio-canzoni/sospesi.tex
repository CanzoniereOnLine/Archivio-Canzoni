%titolo{Sospesi}
%autore{Francesco Camin}
%album{Aria fresca}
%tonalita{RE}
%famiglia{Altre}
%gruppo{}
%momenti{}
%identificatore{sospesi}
%data_revisione{2020_04_23}
%trascrittore{Francesco Endrici}
\beginsong{Sospesi}[by={Camin}]
\beginverse
\[D] Ci son cose che mi domando ma \[B-]non rispondo  \[G]
\[D] cose strane, milioni di rane in \[B-]sottofondo \[G]
\[D] dubbi scuri, che sono muri, che non so ab\[B-]battere  
\[G]senza \[D]te. \[B-] \[G]
\[D] Mi domando perché d'inverno il ramo è \[B-]spoglio \[G]
\[D] e d'estate le sue giornate le veste al \[B-]meglio. \[G]
\[A] Tu accarez\[B-]zavi e \[G]poi
\[A] non ti stan\[B-]cavi \[G]mai.
\endverse
\beginchorus
E adesso ^come fai ^^
a disegnare ^noi ^^
colori non ne ^vuoi ^^
è troppo tardi or^mai. ^^
\endchorus
\beginverse
Cosa ^spinge quel fiore bianco a morire in ^frutto, ^
tu sorri^devi e me lo spiegavi sdraiati ^su un tetto. ^
^ Ma ora se dico noi, la sua eco è un ^mai. ^
\endverse
\beginchorus
E adesso come ^fai ^^
a disegnare ^noi ^^
colori non ne ^vuoi ^^
è troppo tardi or^mai. ^^
\endchorus
\beginverse
Ri\[D]sposte perdute come il \[A]sole d'autunno
come \[B-]foglie cadute sull'a\[G]more e il suo inganno 
come \[D]neve in aprile, come i \[A]giochi in cortile 
come un \[B-]treno che parte ma non \[G]sa dove andare 
e ora \[D]sono distanti gli i\[A]stanti tra noi
per \[B-]scoprire chi sei e poi ca\[G]pire che è troppo tardi or\[D]mai. \[G]
\endverse
\beginchorus
E adesso come ^fai ^^
a disegnare ^noi ^^
colori non ne ^vuoi ^^
è troppo tardi or^mai. ^^
\endchorus
\endsong
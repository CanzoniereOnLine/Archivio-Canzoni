%titolo{La caccia di Kaa}
%autore{}
%album{}
%tonalita{Mi-}
%famiglia{Scout}
%gruppo{}
%momenti{}
%identificatore{la_caccia_di_kaa}
%data_revisione{2012_06_23}
%trascrittore{Francesco Endrici}
\beginsong{La caccia di Kaa}
\beginverse
Nella \[E-]giungla silen\[B7]ziosa \brk Mowgli dorme e si ri\[E-]posa
ma dagli \[A-]alberi più \[E-]alti \brk sono \[B7]scese con due \[E-]salti \[A-]
le invidiose \[E-]Bandarlog \[A-] \brk ecco Mowgli \[E-]pren\[B7]do\[E-]no.
\itshape \[C]Chil se\[D]gui la \[G]traccia \[E-]tu \brk \[C]ed informa \[D]poi Ba\[E-]loo. \rep{2}
\endverse
\beginverse
\chordsoff
Baloo l'^orso si di^spera \brk con Bagheera la pan^tera;
l'avvol^toio dà la ^traccia \brk e si ^parte per la ^caccia. ^
Alle Tane ^Fredde, orsù, ^ \brk Kaa, Bagheera ^e ^Ba^loo.

\itshape^Illo, ^Illo, ^guarda ^su \brk la ^traccia porta a ^te, \brk Ba^loo. \rep{2}
\endverse
\beginverse
\chordsoff
Dolo^rante ed affa^mato \brk dalle scimmie circon^dato,
Mowgli ^tenta di scap^pare, \brk alla sua ^giungla vuol tor^nare; ^
ma le scimmie ac^corrono ^ \brk indietro lo ri^por^ta^no.

\itshape^Nella ^Giungla ^forte e ^buon \brk ^come il Bandar^log,\brk nes^sun. \rep{2}
\endverse
\beginverse
\chordsoff
Senza in^dugio or Ba^gheera \brk mostra tutta la sua ^ira;
la zampa ^di Baloo è pe^sante \brk ma le ^scimmie son ^tante; ^
sol terrore han^no di Kaa: ^ \brk la sua Caccia è ^que^sta ^qua!

\itshape^Striscia, ^danza e ^can^ta \brk ^la sua fame è ^tan^ta! \rep{2}
\endverse
\beginverse
\chordsoff
La Legge ^della giungla im^pone \brk una giusta puni^zione:
Mowgli ^ha disobbe^dito \brk e di ^questo si è pen^tito; ^
le busse quindi ac^cetterà, ^ \brk poi finisce ^tut^to ^qua.

\itshape La ^puni^zion che ^meri^ti \brk can^cella tutti i ^debi^ti. \rep{2}
\endverse
\endsong


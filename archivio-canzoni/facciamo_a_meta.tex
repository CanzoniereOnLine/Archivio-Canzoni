%titolo{Facciamo a metà}
%autore{Eugenio in Via di Gioia}
%album{}
%tonalita{SOL}
%famiglia{Altre}
%gruppo{}
%momenti{}
%identificatore{facciamo_a_meta}
%data_revisione{2025_12_10}
%trascrittore{Francesco Endrici}
\beginsong{Facciamo a metà}[by={Eugenio in Via di Gioia}]
\capo{3}
\beginverse
\[G]Non sarò mai pettinato, puntuale
\[E-]non sarai mai grande e forte per tuo padre,
\[A-]ma se c'è una bomba dentro casa \[D]noi \brk tagliamo insieme il filo ver\[G]de
\[G]cambiamo il rubinetto quando perde
non mi spaventa niente con \[C]te.
Quando pioverà davanti a \[E-]noi,  \brk salteremo il \[D]fango insieme.
\[C] Non parlo mai d'amore, ma con te mi \[D]viene
\endverse
\beginchorus
\[G]Se ti guardo penso che per ogni
\[E-]passo, pianto, inverno, ci sei sempre.
\[C] Non so cos'è la felicità, ma se vuoi facciamo a me\[D]{tà.}
\[G] Se mi guardi sento che mi hai dato
un \[E-]posto un pugno, un senso tra la gente.
\[C] Non so cos'è la felicità, ma se vuoi facciamo a me\[D]{tà.}
\endchorus
\beginverse
^Non sarò per te quello che vince un orso gigante
^non sarai mai sveglia se passa una stella cadente
^ma c'è chi rischia tutto, invece ^noi \brk passiamo solo con il ^verde
^leggiamo gli ingredienti e le etichette ^
non ci spaventa niente. 
\endverse
\beginchorus
\[G]Se ti guardo penso che per ogni
\[E-]passo, pianto, inverno, ci sei sempre.
\[C] Non so cos'è la felicità, ma se vuoi facciamo a me\[D]{tà.}
\[G] Se mi guardi sento che mi hai dato
un \[E-]posto un pugno, un senso tra la gente.
\[C] Non so cos'è la felicità, ma se vuoi facciamo a me\[D]{tà.}
\endchorus
\beginverse
\[E-]Delle notti, dei giorni, degli incubi e i sogni
facciamo a me\[D]tà
\[C]delle multe, dei conti, dei mostri e dei dolci
facciamo a metà. \[A-]
E poi buttiamoci come i ve\[D]stiti
e poi stendiamoci come len\[G]zuola
compagni di banco il primo giorno di \[E-]scuola
consolami se \[C]puoi
e fammi ridere tu che ci \[D]riesci.
\endverse
\beginchorus
\[G]Se ti guardo penso che per ogni
\[E-]passo, pianto, inverno, ci sei sempre.
\[C] Non so cos'è la felicità, ma se vuoi facciamo a me\[D]{tà.}
\[G] Se mi guardi sento che mi hai dato
un \[E-]posto un pugno, un senso tra la gente.
\[C] Non so cos'è la felicità, ma se vuoi facciamo a me\[D]{tà.}
\endchorus
\endsong

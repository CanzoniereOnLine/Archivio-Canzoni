%titolo{Dolcenera}
%autore{Fabrizio De André}
%album{Anime salve}
%tonalita{SI-}
%famiglia{Altre}
%gruppo{}
%momenti{}
%identificatore{dolcenera}
%data_revisione{2020_03_10}
%trascrittore{Francesca Alampi}
\beginsong{Dolcenera}[by={De André}]
\beginverse*
\[B-]Amìala ch'a l'aria cum'a l'è cum'a l'è
Amìala ch'a l'ariach'a l'è le ch'a l'è le
Amìala ch'a l'aria cum'a l'è cum'a l'è
Amìala ch'a l'ariach'a l'è le ch'a l'è le
\endverse
\beginverse
Nera che porta \[E-]via che porta via la \[B-]via
nera che non si ve\[F#7]deva da una vita intera
così dolce\[B-]nera 
nera che picchia \[E-]forte che butta giù le \[B-]porte
nu l'è l'aegua ch'à fà \[F#7]baggià
imbaggià \[B-]imbaggià.
\endverse
\beginverse
Nera di mala\[E-]sorte che ammazza e passa\[B-] oltre
nera come la sfor\[F#7]tuna che si fa la tana
dove non c'è \[B-]luna 
nera di falde a\[E-]mare che passano le \[B-]bare
atru da st\[F#7]ramua a nu n'à a nu\[B-] n'à
\endverse
\beginchorus
Ma la moglie di An\[F#]selmo non lo deve \[B-]sapere
che è venuta per \[F#]me
è arrivata da \[B-]un'ora 
e l'amore ha l'\[E-]amore come \[F#]solo argo\[B-]mento 
e il tumulto del \[F#]cielo ha sbagliato mo\[B-]mento.
\endchorus
\ifchorded
\beginverse*
\vspace*{-\versesep}
{\nolyrics \[F#7]    \[B-]   \[F#7]    \[B-]   \[F#]   \[B-]   \[F#]   \[B-]  }
\endverse
\fi
\beginverse
Acqua che non si \[E-]aspetta altro che bene\[B-]detta
acqua che porta \[F#7]male sale dalle scale 
sale senza \[B-]sale 
acqua che spacca il \[E-]monte che affonda terra e \[B-]ponte
nu l'è l'aegua de 'na ramma
\[F#7]'n calabà 'n \[B-]calabà 
\endverse
\beginchorus 
Ma la moglie di An\[F#]selmo sta sognando del \[B-]mare
quando ingorga gli an\[F#]fratti si ritira e ri\[B-]sale
e il lenzuolo si \[E-]gonfia sul \[F#]cavo dell'\[B-]onda
e la lotta si \[C#]fa scivol\[F#]osa e pro\[B-]fonda.
\endchorus
\ifchorded
\beginverse*
\vspace*{-\versesep}
{\nolyrics \[F#]   \[B-]   \[B7]   \[E-]   \[B-]   \[C#-]    \[F#]   \[B-]   \[B7]   \[E-]   \[B-]   \[F#]  }
\endverse
\fi
\beginverse
\[B-]Amìala ch'a l'aria cum'a l'è cum'a l'è
Amìala ch'a l'ariach'a l'è le ch'a l'è le
acqua di spilli \[E-]fitti dal cielo e dai so\[B-]ffitti
acqua per fotogra\[F#7]fie per cercare i complici da male\[B-]dire
acqua che stringe i \[E-]fianchi tonnara di pa\[B-]ssanti
atru da camalla    \[F#7]   
a nu n'à a nu n'à   \[B-]  
\endverse
\beginverse 
Oltre il muro dei \[F#]vetri si risveglia la \[B-]vita
che si prende per \[F#]mano a battaglia \[B-]finita
come fa questo a\[E-]more che dall'\[F#]ansia di \[B-]perdersi
ha avuto in un \[C#]giorno la ce\[F#]rtezza di a\[B-]versi.
\endverse
\ifchorded
\beginverse*
\vspace*{-\versesep}
{\nolyrics \[F#7]    \[B-]   \[F#7]    \[B-]   \[F#]   \[B-]   \[F#]   \[B-]  }
\endverse
\fi
\beginverse
Acqua che ha fatto \[E-]sera che adesso si ri\[B-]tira
bassa sfila tra la \[F#7]gente come un'innocente
che non c'entra \[B-]niente
fredda come un \[E-]dolore Dolcenera \[B-]senza cuore 
atru da camalla
a nu n\[F#7]'à a nu n'\[B-]à 
\endverse
\beginchorus  
E la moglie di An\[F#7]selmo sente l'acqua che \[B-]scende
dai vestiti incol\[F#7]lati da ogni gelo di \[B-]pelle
nel suo tram scolle\[E-]gato da \[F#]ogni di\[B-]stanza
nel bel mezzo del \[C#]tempo che a\[F#]desso le a\[B-]vanza
così fu quell'a\[E-]more dal man\[F#]cato fi\[B-]nale
così splendido e v\[C#]ero da po\[F#]tervi ingan\[A#]nare.
\endchorus
\ifchorded
\beginverse*
\vspace*{-\versesep}
{\nolyrics \[A]  \[G#]   \[G]  \[F#]   \[B-]   \[F#]   \[B-]   \[B7]   \[E-]   \[B-]   \[C#-]    \[F#]   \[B-]   \[F#]   \[B-]   \[B7]   \[E-]   \[B-]   \[F#]  }
\endverse
\fi
\beginverse
\[B-]Amìala ch'a l'aria cum'a l'è cum'a l'è
Amìala ch'a l'ariach'a l'è le ch'a l'è le
Amìala ch'a l'aria cum'a l'è cum'a l'è
Amìala ch'a l'ariach'a l'è le ch'a l'è le
\endverse
\endsong

%titolo{Giù dai colli}
%autore{Secondo Rastello, Michele Gregorio}
%album{}
%tonalita{RE}
%famiglia{Altre}
%gruppo{}
%momenti{}
%identificatore{giu_dai_colli}
%data_revisione{2020_01_28}
%trascrittore{Fabrizio Gonzo}
%video{}
\beginsong{Giù dai colli}[by={Rastello, Gregorio}]
\beginverse
Giù dai \[D]colli un \[A]dì lon\[D]tano
con la \[G]sola madre ac\[D]can\[A7]to
sei ve\[D]nuto a \[G]questo\[D] piano
dei tuoi \[F#-]sogni al \[C#]dolce in\[F#7]canto.
Ora, o \[E-]Pa\[A7]dre, non più \[D]so\[B-]lo
per le \[E-]stra\[A7]de passi an\[D]cora,
di tuoi figli immenso s\[G]tuolo\[A]
con gran \[D]giubi\[G]lo ti o\[A7]no\[D]ra.
\endverse

\beginchorus
\[D]Don Bosco ri\[G]torna\[A7]
tra i giovani an\[D]cor,\[B-]
ti chiaman fre\[E-]menti
di \[A7]gioia e d’a\[D]mor. \rep{2}
\endchorus
\beginverse
Sul tuo ^colle a^ppare, o ^Santo,
la ca^setta di fa^mi^glia.
Mera^viglia: or ^vedi ac^canto
grande ^tempio, ^grande al^tare.
Ci ri^co^rda il tuo na^ta^le,
i tuoi ^so^gni, il tuo la^voro.
La sua guglia in alto ^sale,^
custo^disce un ^gran te^so^ro.
\endverse

\beginchorus
\[D]Don Bosco ri\[G]torna\[A7]
tra i giovani an\[D]cor,\[B-]
ti chiaman fre\[E-]menti
di \[A7]gioia e d’a\[D]mor. \rep{2}
\endchorus
\beginverse
Da ogni ^parte o^sserva, o ^Padre,
quanti ^giovani in pre^ghie^ra.
Tu li af^fidi a ^dolce ^Madre
perché o^gnuno ^arrivi a ^sera.
Oltre i ^ma^ri, oltre i ^mo^nti
t’invo^chia^mo, Padre ^santo.
Fino agli ultimi oriz^zonti^
lieto e^cheggia il ^nostro ^ca^nto.
\endverse

\beginchorus
\[D]Don Bosco ri\[G]torna\[A7]
tra i giovani an\[D]cor,\[B-]
ti chiaman fre\[E-]menti
di \[A7]gioia e d’a\[D]mor. \rep{2}
\endchorus


\endsong

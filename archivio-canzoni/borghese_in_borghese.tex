%titolo{Borghese in borghese}
%autore{Fulminacci}
%album{}
%tonalita{LA}
%famiglia{Altre}
%gruppo{}
%momenti{}
%identificatore{borghese_in_borghese}
%data_revisione{2020_03_28}
%trascrittore{Francesca Alampi}
\beginsong{Borghese in borghese}[by={Fulminacci}]
\beginverse
\[A]Mi stranisce questo fatto che tutti sono bipolari
o quanto meno si dichiarino tali, che poi
\[D7]è troppo facile dire che c'hai problemi strani
\[A]spesso ad aiutarti è chi c'ha i problemi reali.
Quegli esseri normali, che leggono i giornali
quelli con una moglie, un lavoro, un figlio e due cani.
\[A]Quelli che\[D7] la notte dormono, non si confondono
\[D7]che se gli par\[A]li ti rispondono, gli esseri umani.
\[A]E stanno bene, lo sai perché stanno bene?
Perché ogni giorno loro accettano la propria natura.
Loro non fanno \[D7]come te
che cerchi sempre una cura.
\[D7]Loro non hanno \[A]né tempo
\[A]né voglia di avere paura
\[E7]e figuriamoci se nella loro vita
c'è spazio per un proble\[D7]ma
\[D7]che non riguardi la Roma o la Lazio.
\[A]Per un disturbo, un dilemma un disagio da pazzo
che non ha capo, né coda, né braccia (né cazzo).
\endverse
\beginchorus 
\[A]Io canto la mia opinione così che si diffonda,
sono un borghese in borghese
è così che mi nascondo.
\[A]Sono una statua \[D7]di bronzo
\[D7]così che possa fondermi
con la tua faccia \[A]da stronzo
\[A]che uso per difendermi. \rep{2}
\endchorus
\beginverse 
\[A]Io per difendermi non faccio mai niente\chordsoff
a quantità di calorie che brucia la gente.
Tu non ci pensi, ma se guardi la tua vita
o anche se guardi la mia dall'alto
resti sempre un perdente
resto sempre un perdente.
Sono vent'anni che ci sono
ma non sono nessuno.
Sono dieci anni che suono
sono tre anni che fumo
\chordson\[D7]sono tre giorni che ho sonno
sono tre docce che sudo.
\[A]Questo lo so però non so
se ho messo il sale nel sugo
sono sicuro di sì, però non voglio assaggiare.
Ora il problema è riuscire
prima di uscire a cagare.
\[D7]Che poi il problema più grande di tutti
\[D7]a repressione dei ret\[A]ti con la censura dei rutti.
\[E7]Ma le stazioni dei treni
\[E7]non mi commuovono più
\[E7]nei nostri comodi \[D7]luoghi comuni mi confortano
\[A]queste opinioni a metà
che valgono sempre di più
le costosissime Polaroid che ritornano.
\endverse
\beginchorus
\[A]Io canto la mia opinione così che si diffonda,
sono un borghese in borghese
è così che mi nascondo.
\[A]Sono una statua \[D7]di bronzo
\[D7]così che possa fondermi
con la tua faccia \[A]da stronzo
\[A]che uso per difendermi. \rep{2}
\endchorus
\beginverse
Denominazione di origine controllata,
so de periferia però non quella inflazionata.
\[D7]Tu stringi un patto col diavolo
io aspetto il patto con l'ATAC.
Il sabato fuori all'organico
sono schiavo dell'AMA.
Dalle mie parti sono più buche che asfalto
e ogni tanto i lampioni si accendono come d'incanto.
\[D7]E pare si stia diffondendo un'epidemia
\[A]nota al quartiere col nome di Idrofollia.
Se non ci fosse quel magico mondo di funghi e infezioni
\[A]avrei serie difficoltà a dare le indicazioni
grande raccordo anulare, uscita 33
solo se vieni dal mare, passami a prendere.
\[E7]Tra vent'anni dicono ci sarà la rivoluzione
\[D7]dove passa il treno ci costruiscono la stazione.
\[A]Già mi immagino la gente per strada
che è pronta a urlare
"Rivogliamo la libertà di poterci lamentare".
\endverse
\beginchorus 
\[A]Io canto la mia opinione così che si diffonda,
sono un borghese in borghese
è così che mi nascondo.
\[A]Sono una statua \[D7]di bronzo
\[D7]così che possa fondermi
con la tua faccia \[A]da stronzo
\[A]che uso per difendermi.
\[A]Io canto la mia opinione così che si diffonda,
sono un borghese in borghese
è così che mi nascondo.
\[A]Sono una statua \[D7]di bronzo
\[D7]così che possa fondermi
con la tua faccia \[A]da stronzo
\[A]che uso per difendermi.
\[A]Io canto la mia opinione così che si diffonda
\[D7]sono un borghese in borghese
è così che mi nascondo.
\[A]Io canto la mia opinione così che si diffonda
\[D]sono un borghese in borghese
è così che mi nascondo.
(Io canto ma non è che, non è che canto proprio eh).
\endchorus
\endsong

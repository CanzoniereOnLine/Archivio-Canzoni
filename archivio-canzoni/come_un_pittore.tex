%titolo{Come un pittore}
%autore{Modà}
%album{}
%tonalita{LA}
%famiglia{altre}
%gruppo{}
%momenti{}
%identificatore{come_un_pittore}
%data_revisione{}
%trascrittore{Francesco Raccanello}
\beginsong{Come un pittore}[by={Modà}]

\beginverse
\[A]Ciao, sem\[E]plicemente \[F#-]ciao,
\[E]Difficile tro\[D]var parole \[A]molto serie,
\[D]tenterò di \[E]disegnare
\[A]Co\[E]me un pitto\[F#-]re,
\[E]Farò in modo di \[D]arrivare \[A]dritto al cuore
\[D]con la forza \[E]del colore
\[D]guarda...\[E]Senza parlare
\[F#-]azzurro come \[D]te, come il cielo \[A]e
il ma\[F#-]re
\[D]E giallo come \[E]luce del \[A]sole
\[E]Rosso come \[F#-]le  co\[E]se che mi \[D]fai \[E] provare...
\endverse

\beginverse
^Ciao, sem^plicemente ^ciao,
^disegno l’erba ^verde come ^la speranza
^e come frutta an^cora acerba
^E a^desso un po' di ^blu come la ^notte
^E bianco come ^le sue stelle
^con le  sf^umature gialle
^E l’aria ^ puoi solo respirarla
^Azzurro come ^te, come il cielo ^e
il ma^re
^E giallo come ^luce del ^sole..
^Rosso come ^le  co^se che mi ^fai ^  provare...
\endverse

\beginverse
\[D]Per le tem\[A]peste non ho il colore	
\[D]con quel che \[A]resta disegno un fiore
\[D]ora che è estate, \[E]ora che è amore.
\[F#-]Azzurro come \[D]te, come il cielo \[A]e
il ma\[F#-]re
\[D]E giallo come \[E]luce del \[A]sole
\[E]Rosso come \[F#-]le  co\[E]se che mi \[D]fai \[E] provare...
\endverse

\endsong

%titolo{Andrea}
%autore{Fabrizio De André}
%album{}
%tonalita{RE}
%famiglia{altre}
%gruppo{}
%momenti{}
%identificatore{andrea_se_perso}
%data_revisione{}
%trascrittore{Francesco Raccanello}
\beginsong{Andrea}[by={De André}]

\beginverse
\[D]Andrea s'è perso, \brk s'è \[C]perso  e non sa tor\[G]nare. 
\[D]Andrea s'è perso, \brk s'è \[C]perso  e non sa tor\[G]nare. 
An\[D]drea aveva un'a\[C]more, riccioli \[G]neri. 
An\[D]drea aveva un do\[C]lore, riccioli \[G]neri. 
\endverse

\chordsoff
\beginverse
C'era scritto sul foglio \brk che era morto sulla bandiera. 
C'era scritto, e la firma \brk era d'oro, era firma di re. 
Ucciso sui monti \brk di Trento dalla mitraglia. 
Ucciso sui monti \brk di Trento dalla mitraglia. 
\endverse

\beginverse
Occhi di bosco \brk contadino del regno profilo francese. 
Occhi di bosco \brk soldato del regno profilo francese. 
E Andrea l'ha perso, \brk ha perso l'amore, la perla più rara. 
E Andrea ha in bocca \brk ha in bocca un dolore, la perla più scura. 
\endverse

\beginverse
Andrea coglieva \brk raccoglieva violette ai bordi del pozzo. 
Andrea gettava \brk riccioli neri nel cerchio del pozzo. 
Il secchio gli disse  \brk gli disse: “Signore il pozzo e' profondo, 
più fondo del fondo degli occhi \brk della Notte del Pianto”. 
Lui disse: “Mi basta, \brk mi basta che sia più profondo di me”. 
Lui disse: “Mi basta,  \brk mi basta che sia più profondo di me”. 
\endverse

\endsong

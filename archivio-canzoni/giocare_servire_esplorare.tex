%titolo{Giocare, Servire, Esplorare}
%autore{Elena Terziotti, Dario Campanella}
%album{Convegno Regionale EmiRo 2012}
%tonalita{Fa}
%famiglia{Scout}
%gruppo{}
%momenti{}
%identificatore{giocare_servire_esplorare}
%data_revisione{2012_08_29}
%trascrittore{Francesco Endrici}
\beginsong{Giocare, Servire, Esplorare}[by={Terziotti, Campanella}]
\beginverse
Abbiam sor\[F]riso e guardato lon\[B&]tano
e il nostro \[D-]sguardo ci ha portato fin \[C]qua.
Nel nostro \[F]zaino c'è quello che \[B&]siamo 
c'è quello che è \[D-]stato e \[C]quel che sa\[F]rà.
\endverse
\beginverse
Quando un fra^tello che cammina al mio ^fianco 
mi chiede a^more non mi faccio più in ^là.
So che fi^dandoci l'un l'altro po^tremo 
lasciare un ^segno, cambiar la real\[C]tà!
\endverse
\beginchorus
\[F]Questo è il mio \[C]sogno, \brk colo\[B&]rare di senso il presente guardandomi at\[F]torno.
\[F]Questo è il mio \[C]sogno, \brk i pen\[B&]sieri, le mani, le voci, i colori del \[F]mondo.
\[F]Questo è il mio \[C]sogno, \brk è gri\[B&]dare il tuo amore e cantarlo insieme ogni \[F]giorno.
\endchorus
\beginverse
All'incer^tezza che ogni giorno ci ^sfida 
noi rispon^diamo ancora “Eccomi ^qua”. 
Perché il co^raggio di rischiare vuol ^dire 
far vivere un ^sogno che ^non svani^rà.
\endverse
\beginverse
Non è più il ^tempo di restare a guar^dare 
ma dare ^voce a chi una voce non ^ha. 
Voglio gio^care, servire, esplo^rare 
lasciare un ^segno, cambiar la real\[C]tà.
\endverse
\beginchorus
\[F]Questo è il mio \[C]sogno, \brk colo\[B&]rare di senso il presente guardandomi at\[F]torno.
\[F]Questo è il mio \[C]sogno, \brk i pen\[B&]sieri, le mani, le voci, i colori del \[F]mondo.
\[F]Questo è il mio \[C]sogno, \brk è gri\[B&]dare il tuo amore e cantarlo insieme ogni \[F]giorno.
\endchorus
\endsong


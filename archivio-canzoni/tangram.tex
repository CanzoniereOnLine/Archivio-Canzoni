%titolo{Tangram}
%autore{Favotti}
%album{}
%tonalita{}
%famiglia{Scout}
%gruppo{}
%momenti{}
%identificatore{tangram}
%data_revisione{2023_05_13}
%trascrittore{Francesco Endrici}
\beginsong{Tangram}[by={Favotti}]
\ifchorded
\beginverse*
\vspace*{-0.8\versesep}
{\nolyrics \[D]\[F#-]\[G]\[D]\[D]\[F#-]\[G]\[D]}
\vspace*{-\versesep}
\endverse
\fi
\beginchorus
Dimmi che forma \[D]ha la tua felici\[F#-]tà,
parlami dei suoi \[G]angoli e le curve che lei \[D]fa.
Dimmi se in essa \[D]c'è un posto anche per \[F#-]me,
perché la mia felici\[G]tà comincia anche da \[D]te.
\endchorus
\beginverse \memorize
È un \[A-7]Cerchio il nostro modo \brk di incon\[G]trarci e stare insieme,
ro\[D]tondo il piatto che porgiamo \brk a \[A]chi ci chiede un bene.
Non \[C]c'è un'uscita o angolo, \brk na\[G]scondersi non vale, 
tut\[D]ti protagonisti, \brk spesi \[A]per un ideale.
\endverse
\beginchorus
Dimmi che forma \[D]ha la tua felici\[F#-]tà,
parlami dei suoi \[G]angoli e le curve che lei \[D]fa.
Dimmi se in essa \[D]c'è un posto anche per \[F#-]me,
perché la mia felici\[G]tà comincia anche da \[D]te.
\endchorus
\beginverse
^Triangolo è quel fazzoletto \brk ar^rotolato al collo, 
u^na tendina fragile \brk ma ^che non teme il crollo,
un ^Dio che da lassù \brk ci guarda e in ^Tre si fa per tutti,
nes^suno escluso dal suo amore: \brk i ^belli e i farabutti.
\endverse
\beginchorus
Dimmi che forma \[D]ha la tua felici\[F#-]tà,
parlami dei suoi \[G]angoli e le curve che lei \[D]fa.
Dimmi se in essa \[D]c'è un posto anche per \[F#-]me,
perché la mia felici\[G]tà comincia anche da \[D]te.
\endchorus
\beginverse
Qua^drato è quel momento \brk in cui fac^ciamo in noi memoria,
i ^lati di una piazza \brk dove ^c'è la nostra storia,
ad ^ogni angolo qualcuno \brk ^che vuole giocare,
al ^centro un bene che è comune e \brk ^tutti vuol toccare!
\endverse
\transpose{2}
\beginchorus
Dimmi che forma \[D]ha la tua felici\[F#-]tà,
parlami dei suoi \[G]angoli e le curve che lei \[D]fa.
Dimmi se in essa \[D]c'è un posto anche per \[F#-]me,
perché la mia felici\[G]tà comincia anche da \[D]te.
\endchorus
\beginverse
Il ^Parallelogramma \brk è un aqui^lone per volare,
sul^le periferie, \brk le porte ^chiuse da varcare;
u^na cartina e un azimut \brk ci ^mostrano una traccia,
se o^gnuno esce di casa \brk e poi ci ^mette la sua faccia.
\endverse
\beginchorus
Dimmi che forma \[D]ha la tua felici\[F#-]tà,
parlami dei suoi \[G]angoli e le curve che lei \[D]fa.
Dimmi se in essa \[D]c'è un posto anche per \[F#-]me,
perché la mia felici\[G]tà comincia anche da \[D]te.
\endchorus
\beginverse
Ma ^per giocare al meglio \brk questo ^gioco della vita
e ^fare in modo che \brk sia una par^tita mai finita,
è ^logico, si può giocare \brk ^solamente insieme
per ^generare forme \brk sempre ^nuove e sempre piene.
\endverse
\beginchorus
Dimmi che forma \[D]ha la tua felici\[F#-]tà,
parlami dei suoi \[G]angoli e le curve che lei \[D]fa.
Dimmi se in essa \[D]c'è un posto anche per \[F#-]me,
perché la mia felici\[G]tà comincia anche da \[D]te.
\endchorus
\beginchorus
Dimmi la verità, \brk \[D] dimmi che già si sa, \[F#-]
dimmi che tutti in\[G]sieme \brk noi arriveremo \[D]là,
dimmi come sa\[D]rà \brk e che colore a\[F#-]vrà,
dimmi che sarà \[G]vera e pura\dots{} \brk la felici\[D]tà!
\endchorus
\endsong

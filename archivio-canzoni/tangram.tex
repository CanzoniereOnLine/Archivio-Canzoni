%titolo{Tangram}
%autore{Favotti}
%album{}
%tonalita{}
%famiglia{Scout}
%gruppo{}
%momenti{}
%identificatore{tangram}
%data_revisione{2023_05_13}
%trascrittore{Francesco Endrici}
\beginsong{Tangram}[by={Paolo Favotti}]
\chordsoff
\ifchorded
\beginverse*
\vspace*{-0.8\versesep}
{\nolyrics }
\vspace*{-\versesep}
\endverse
\fi
\beginchorus
Dimmi che forma ha la tua felicità,
parlami dei suoi angoli e le curve che lei fa.
Dimmi se in essa c'è un posto anche per me,
perché la mia felicità comincia anche da te.
\endchorus
\beginverse \memorize
È un Cerchio il nostro modo di incontarci e stare insieme,
rotondo il piatto che porgiamo a chi ci chiede un bene.
Non c'è un'uscita o angolo, nascondersi non vale, 
tutti protagonisti, spesi per un ideale.
\endverse
\beginchorus
Dimmi che forma ha la tua felicità,
parlami dei suoi angoli e le curve che lei fa.
Dimmi se in essa c'è un posto anche per me,
perché la mia felicità comincia anche da te.
\endchorus
\beginverse
Triangolo è quel fazzoletto arrotolato al collo, 
una tendina fragile ma che non teme il crollo,
un Dio che da lassù ci guarda e in Tre si fa per tutti,
nessuno escluso dal suo amore: i belli e i farabutti.
\endverse
\beginchorus
Dimmi che forma ha la tua felicità,
parlami dei suoi angoli e le curve che lei fa.
Dimmi se in essa c'è un posto anche per me,
perché la mia felicità comincia anche da te.
\endchorus
\beginverse
Quadrato è quel momento in cui facciamo in noi memoria,
i lati di una piazza dove c'è la nostra storia,
ad ogni angolo qualcuno che vuole giocare,
al centro un bene che è comune e tutti vuol toccare!
\endverse
\beginchorus
Dimmi che forma ha la tua felicità,
parlami dei suoi angoli e le curve che lei fa.
Dimmi se in essa c'è un posto anche per me,
perché la mia felicità comincia anche da te.
\endchorus
\beginverse
Il Parallelogramma è un aquilone per volare,
sulle periferie, le porte chiuse da varcare;
una cartina e un azimut ci mostrano una traccia,
se ognuno esce di casa e poi ci mette la sua faccia.
\endverse
\beginchorus
Dimmi che forma ha la tua felicità,
parlami dei suoi angoli e le curve che lei fa.
Dimmi se in essa c'è un posto anche per me,
perché la mia felicità comincia anche da te.
\endchorus
\beginverse
Ma per giocare al meglio questo gioco della vita
e fare in modo che sia una partita mai finita,
è logico, si può giocare solamente insieme
per generare forme sempre nuove e sempre piene.
\endverse
\beginchorus
Dimmi che forma ha la tua felicità,
parlami dei suoi angoli e le curve che lei fa.
Dimmi se in essa c'è un posto anche per me,
perché la mia felicità comincia anche da te.
\endchorus
\beginchorus
Dimmi la verità, dimmi che già si sa,
dimmi che tutti insieme noi arriveremo là,
dimmi come sarà e che colore avrà,
dimmi che sarà vera e pura\dots{} la felicità!
\endchorus
\endsong

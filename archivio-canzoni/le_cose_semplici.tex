%titolo{Le cose semplici}
%autore{Francesco Camin}
%album{Palindromi}
%tonalita{MI}
%famiglia{Altre}
%gruppo{}
%momenti{}
%identificatore{le_cose_semplici}
%data_revisione{2020_04_23}
%trascrittore{Francesco Endrici}
\beginsong{Le cose semplici}[by={Camin}]
\ifchorded
\beginverse*
\vspace*{-0.8\versesep}
{\nolyrics  \[E]\[D]\[E]\[D]\[C#-]\[D]\[E]\[D]}
\endverse
\fi
\beginverse\memorize
\[E]Semplici, \[A] mi piaccio\[B]no le cose \[E]semplici  
\[A] lascio da \[B]parte i dubbi am\[C#-]letici,
\[A] le strade \[B]vuote quando \[C#-]piove  
il pendolo che \[D]batte le nove. \replay
^Semplici, ^non come i di^scorsi dei po^litici   
o il per^ché o percome ^dei misteri ^cosmici,   
^ mi basta il ^sole la mat^tina 
e l'acqua calda sulla \[B]schiena.  
\endverse
\beginchorus
\[A]Lascia \[B]perde\[E]re, tu non mi potrai capire 
se il \[A]mondo in\[B]torno a \[E]te lo trasformi in ra\[E7]dice di tre  
mentre a \[A]me basta guar\[B]dare il blu   
io a\[E]doro le canne \[C#-] di bambù
\[A] mi piacciono le cose semplici   
\[B] per questo non mi piaci tu.  
\endchorus
\ifchorded
\beginverse*
\vspace*{-\versesep}
{\nolyrics  \[E]\[D]\[E]\[D]\[C#-]\[D]\[E]\[D]}
\endverse
\fi
\beginverse 
^Semplici, ^ mi piaccio^no le cose ^semplici   
 la bel^lezza e l'one^stà dei piatti ^tipici  
^ il caf^fè che sale ^piano  
il mio vicino, ^da lontano.  \replay
Ip^notici  i gra^nelli della ^polvere che ^volano  
e gli uc^celli la mat^tina che cin^guettano.   
^ Anche a te ^piace cinguet^tare  
ma solo con il cellu\[B]lare.
\endverse 
\beginchorus
\[A]Lascia \[B]perde\[E]re, tu non mi potrai capire 
se il \[A]mondo in\[B]torno a \[E]te lo trasformi in ra\[E7]dice di tre  
mentre a \[A]me basta guar\[B]dare il blu   
io a\[E]doro le canne \[C#-] di bambù
\[A] mi piacciono le cose semplici   
\[B] per questo non mi piaci \[C#-]tu.  No!
\endchorus
\ifchorded
\beginverse*
\vspace*{-\versesep}
{\nolyrics  \[A]\[E]\[B]\[B]\[C#-]\[A]}
\endverse
\fi
\beginverse
^Semplici, ^ mi piaccio^no le cose ^semplici   
ora ^chiudi gli occhi e a^scolta il vento e gli ^alberi  
^ a volte ^parlano per ^ore e cancellano il ru^more.  
\endverse
\ifchorded
\beginverse*
\vspace*{-\versesep}
{\nolyrics  \[A]\[B]\[E]\[E7]}
\endverse
\fi 
\beginchorus
Tu il \[A]mondo in\[B]torno a \[E]te lo trasformi in ra\[E7]dice di tre  
mentre a \[A]me basta guar\[B]dare il blu   
io a\[E]doro le canne \[C#-] di bambù
\[A] mi piacciono le cose semplici   
\[B] per questo non mi piaci tu.  
\endchorus
\endsong
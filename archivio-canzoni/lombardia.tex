%titolo{Lombardia}
%autore{Mercanti di liquore}
%album{}
%tonalita{LA-}
%famiglia{altre}
%gruppo{}
%momenti{}
%identificatore{lombardia}
%data_revisione{}
%trascrittore{Francesco Raccanello}
\beginsong{Lombardia}[by={Mercanti di liquore}]

\beginverse
\[A-]Atterrati “su in Bri\[C]anza” \brk come un \[G]settequattro\[A-]sette
siam cresciuti di nas\[C]costo, \brk come \[G]le castagne \[A-]matte
la re\[F]gina Teodolinda \brk ci fa\[G]ceva l'occhio\[A-]lino
ma noi \[A-]irricono\[D]scenti, \brk non gli ab\[F]biam \[G]fatto l'in\[A-]chino.
Impa\[A-]rammo la chi\[C]tarra \brk per a\[G]vere un occa\[A-]sione
per paura di sen\[C]tirci \brk come un \[G]mobile a Lis\[A-]sone
poi ci \[F]siamo travestiti \brk da sol\[G]dati di ven\[A-]tura
per cercare di sca\[D]lare \brk questa \[F]ripi\[G]da pia\[A-]nura
\endverse

\beginchorus
\[A-]Lombar\[G]dia, \brk com'è \[C]facile volerti \[G]male
di sor\[A-]risi non ne \[G]fai \brk e ti \[C]piace maltrat\[G]tare
ma noi \[C]siamo i figli \[G]storti, \brk nati \[F]dentro un'oste\[A-]ria
e riusciamo a respi\[G]rare, \brk pur es\[F]sendo in \[G]Lombar\[A-]dia
\endchorus

\chordsoff
\beginverse
A Milano costruimmo \brk una giostra di cristallo
ma la pioggia di monete \brk l'ha distrutta sul più bello
riparammo nei quartieri \brk dove c'è periferia
perché sotto l'immondizia \brk sta nascosta la magia

E fu notte sempre lunga, \brk ubriaca nei sobborghi
imparammo a camminare \brk con il passo dei balordi
il profumo dell'asfalto \brk ed il nome dei coltelli
diventammo spazzatura, \brk diventammo molto belli. 
\textnote{Rit.}
\endverse

\beginverse
Quando venne l'uragano \brk ci sorprese sopra Lecco
lo prendemmo per la coda \brk e lo ficcammo dentro al sacco
anche il lago fu gentile, \brk ci ha svelato il suo mistero
con in cambio la promessa \brk di non raccontarlo in giro
Abbiam preso qualche stella \brk dalla notte bergamasca
mentre il diavolo rideva \brk gli fregammo la sua crusca
poi chiedemmo alla montagna \brk di cantarci una canzone
e nella valle sottostante \brk tutti fecero l'amore
\textnote{Rit.}
\endverse

\beginverse
Abbiam fatto la scommessa \brk di una vita rattoppata
come quando giochi il due \brk nella briscola chiamata
non ci provoca vergogna \brk la volgarità o il baccano
perché anche l’occhio pesto \brk può vedere assai lontano
Quindi non ci biasimare \brk se non siamo riverenti
È difficile parlare \brk con in bocca il paradenti
Se non puoi volerci bene \brk facci almeno compagnia
Tanto sai dove trovarci\dots \brk buonanotte Lombardia
\textnote{Rit.}
\endverse

\endsong

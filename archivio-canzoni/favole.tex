%titolo{Favole}
%autore{Alessandro Intonti}
%album{}
%tonalita{Do}
%famiglia{Scout}
%gruppo{}
%momenti{}
%identificatore{favole}
%data_revisione{D2018_09_26}
%trascrittore{Federico Crippa}
%video{https://www.youtube.com/watch?v=MCI9mm_gqW8}
\beginsong{Favole}[by={Intonti}]
\beginverse 
Quante \[C]volte siamo stati con gli \[E-]occhi spalancati a sen\[A-]tire \[C]
\[F] favole incantate o \[C]storie strane di maghi e \[G]fate. \[G7]
Chi di \[C]noi non è mai stato ra\[E-]pito dalle storie ascolta\[A-]te \[C]
e viag\[F]giato sulle strade incre\[C]dibili immagi\[G]nate.\[G7]
\[A-]Favole, storie dove è \[D]facile volare via,
\[F]sopra una nuvola con le \[C]ali della fanta\[G]sia.\[G7] 
\endverse 

\beginverse
Un ab\[C]braccio sincero a un a\[E-]mico che parte e che \[A-]sa \[C] \echo{che prima o poi ci si rincontrerà}
per ri\[F]vivere insieme fe\[C]lici momenti che \[G]poi \[G7] \echo{non potremo più scordare.}
Dor\[C]mire sotto le tende e sotto\[E-]voce stare fermi a parla\[A-]re \[C] \echo{di mille altre avventure.}
\[F]Favole che se vo\[C]gliamo noi possiamo sen\[G]tire \[G7] \echo{se si incomincia solo a\dots}
\endverse
\beginverse 
\[C]Credere, credere, \[E-]credere, credere \[A-]che \[C]
basta po\[F]co sai, anche un \[C]canto e poi tu vedr\[G]ai
\[C]che crescere, vivere, \[E-]ridere noi insieme \[A-]noi \[C]
è fa\[F]vola e in più se \[C]canti anche tu ci sa\[G]rà
la \[A-]voglia di stare stasera
tutti u\[D]niti in questa storia vera
finché il \[F]giorno si prende la notte
per consegnar\[C]la ad una stella na\[G]scente.
\endverse 

\endsong 

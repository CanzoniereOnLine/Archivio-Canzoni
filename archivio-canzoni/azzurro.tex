%titolo{Azzurro}
%autore{Adriano Celentano}
%album{}
%tonalita{RE-}
%famiglia{Altre}
%gruppo{}
%momenti{}
%identificatore{azzurro}
%data_revisione{2020_03_15}
%trascrittore{Francesca Alampi}
\beginsong{Azzurro}[by={celentano}]
\ifchorded
\beginverse*
{\nolyrics \[D-]\[A7]\[D-]\[A7]\[D-]}
\endverse
\fi
\beginverse\memorize
\[D-]Cerco l'e\[A7]state tutto \[D-]l'anno \[A7] \brk e all'improv\[D-]viso \[A7] eccola \[D-]qua         
\[G-]Lei è par\[D7]tita per le \[G-]spiagge \[D7] e sono \[G-]solo \[D7] \brk quassù in cit\[G-]tà 
\[D]sento fi\[A7]schiare sora i \[F#-7]tetti \[B7] \brk un aero\[E7]plano \[A7] che se ne \[D]va. \[A7]
\endverse    
\beginchorus
Az\[D]zurro, il pomeriggio è troppo az\[E7]zurro e \[A7]lungo, \brk per \[E-7]me       \[A7]
mi accorgo di non a\[D]vere più ri\[A7]sorse \brk  \[D]senza \[A7] di \[D]te.      \[D7]   
E al\[G]lora io quasi \[F#-]quasi prendo il tre\[B-]no \brk e \[G]vengo, \[B7] vengo da \[E7]te \[A7]   
ma il \[D]treno dei desi\[B-]deri \[G-] \brk nei miei pen\[D]sieri all'incon\[G]tra\[A7]rio \[D]va. 
\endchorus 
\beginverse
^Cerco un po' d'^Africa in giar^dino,  ^  \brk tra l'ole^andro ^ e il bao^bab
^come fa^cevo da bam^bino, ^  \brk ma qui c'è ^gente, ^ non si può ^più.
^Stanno innaf^fiando le tue ^rose, ^  \brk non c'è il le^one, ^ chissà do^v'è. ^
\endverse 
\beginchorus
Az\[D]zurro, il pomeriggio è troppo az\[E7]zurro e \[A7]lungo, \brk per \[E-7]me       \[A7]
mi accorgo di non a\[D]vere più ri\[A7]sorse \brk  \[D]senza \[A7] di \[D]te.      \[D7]   
E al\[G]lora io quasi \[F#-]quasi prendo il tre\[B-]no \brk e \[G]vengo, \[B7] vengo da \[E7]te \[A7]   
ma il \[D]treno dei desi\[B-]deri \[G-] \brk nei miei pen\[D]sieri all'incon\[G]tra\[A7]rio \[D]va.  
\endchorus 
\beginverse
^Sembra quan^d'ero all'ora^torio, ^  \brk con tanto ^sole, ^ tanti anni ^fa
^quelle do^meniche da ^solo ^  \brk in un cor^tile, ^ a passeg^giar,
^ora mi an^noio più di al^lora, ^  \brk neanche un ^prete ^ per chiacchie^rar. ^
\endverse 
\beginchorus
Az\[D]zurro, il pomeriggio è troppo az\[E7]zurro e \[A7]lungo, \brk per \[E-7]me       \[A7]
mi accorgo di non a\[D]vere più ri\[A7]sorse \brk  \[D]senza \[A7] di \[D]te.      \[D7]   
E al\[G]lora io quasi \[F#-]quasi prendo il tre\[B-]no \brk e \[G]vengo, \[B7] vengo da \[E7]te \[A7]   
ma il \[D]treno dei desi\[B-]deri \[G-] \brk nei miei pen\[D]sieri all'incon\[G]tra\[A7]rio \[D]va.  
\endchorus
\endsong
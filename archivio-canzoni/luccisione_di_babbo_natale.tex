%titolo{L'uccisione di Babbo Natale}
%autore{Francesco De Gregori}
%album{}
%tonalita{SOL}
%famiglia{altre}
%gruppo{}
%momenti{}
%identificatore{luccisione_di_babbo_natale}
%data_revisione{}
%trascrittore{Francesco Raccanello}
\beginsong{L'uccisione di Babbo Natale}[by={De Gregori}]
	
\beginverse
\[G]Dolly del mare pro\[C]fondo, \[G] 
figlia di mina\[C]tori, \[G] 
si \[A-]leva le scarpe e cam\[G]mina sull'\[C]erba
insieme al \[G]figlio del \[C]figlio dei \[G]fiori.
\endverse

\chordsoff
\begin{verse}
^E fanno la solita ^strada ^
fino al cada^vere del grillo, ^
la ^luna impaurita li ^guarda pas^sare 
e le ^stelle sono ^punte di ^spillo.
\end{verse}

\begin{verse}
^E mentre le lancette cam^minano ^
i due si di^vidono il fungo ^
e ^intanto mangiando in^gannano il ^tempo 
ma non do^vranno ingan^narlo a ^lungo.
\end{verse}

\begin{verse}
^Infatti arriva Babbo Na^tale, ^
carico di ferro e car^bone, ^
il ^figlio del figlio dei ^fiori lo uc^cide 
con un ^coltello e con ^un bas^tone.
\end{verse}

\begin{verse}
^E Dolly gli pulisce le ^mani ^
con una fetta di ^pane,  ^
le ^nuvole passano ^dietro la ^luna e
da lon^tano sta abba^iando un ^cane.
\end{verse}

\begin{verse}
^E la neve comincia a ca^dere, ^
la neve che ca^deva sul prato ^
e in ^pochi minuti si ^sparse la ^voce 
che Babbo ^Natale era ^stato ammaz^zato.
\end{verse}

\begin{verse}
^Così Dolly del mare pro^fondo ^
e il figlio del figlio dei ^fiori ^ 
si ^danno la mano e ^ritornano a ^casa, 
tornano a ^casa dai ^geni^tori.
\end{verse}

\endsong

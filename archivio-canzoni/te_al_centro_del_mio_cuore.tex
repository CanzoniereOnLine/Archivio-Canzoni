%titolo{Te al centro del mio cuore}
%autore{Gen Verde}
%album{È bello lodarti}
%tonalita{Mi-}
%famiglia{Liturgica}
%gruppo{}
%momenti{Comunione}
%identificatore{te_al_centro_del_mio_cuore}
%data_revisione{2011_12_31}
%trascrittore{Francesco Endrici}
\beginsong{Te al centro del mio cuore}[by={Gen\ Verde}]
\beginverse
\[E-]Ho bisogno di incontrarti nel mio \[G]cuore,
\[B-]di trovare Te, di stare insieme a \[C7+]Te:
\[A-]unico riferimento del mio an\[E-]dare,
\[C]unica ragione \[D]Tu, \[B-]unico sostegno \[E-]Tu.
Al \[C]centro del mio cuore \[D]ci sei solo \[G]Tu.
\endverse
\beginverse
%\chordsoff
^Anche il cielo gira intorno e non ha ^pace,
^ma c'è un punto fermo è quella stella ^là.
^La stella polare è fissa ed è la ^sola,
^la stella polare ^Tu, ^la stella sicura ^Tu.
Al ^centro del mio cuore ^ci sei solo ^Tu.
\endverse
\beginchorus
\[G]Tutto \[B-]ruota intorno a \[C]Te, in funzione di \[B-]Te, \[E-]
e poi \[B-]non importa il \[C]come, il dove e il \[D4]se. \[D]
\endchorus
\beginverse
%\chordsoff
^Che Tu splenda sempre al centro del mio ^cuore,
^il significato allora sarai ^Tu,
^quello che farò sarà soltanto a^more.
^Unico sostegno ^Tu, ^la stella polare ^Tu.
Al ^centro del mio cuore ^ci sei solo ^Tu.
\endverse
\endsong




%titolo{I cento passi}
%autore{Modena City Ramblers}
%album{¡Viva la vida, muera la muerte!}
%tonalita{MI-}
%famiglia{Altre}
%gruppo{}
%momenti{}
%identificatore{i_cento_passi}
%data_revisione{2020_03_10}
%trascrittore{Francesca Alampi}
\beginsong{I cento passi}[by={Modena City Ramblers}]
\ifchorded
\beginverse*
\vspace*{-0.8\versesep}
{\nolyrics \[E-] \[G] \[C] \[B] }
\endverse
\fi
\beginverse
\[E-]Nato nella terra dei \[G]vespri e degli aranci
tra \[C]Cinisi e Palermo par\[B]lava alla sua radio.
Negli \[E-]occhi si leggeva la \[G]voglia di cambiare
la \[C]voglia di Giustizia che \[B]lo portò a lottare.
A\[E-]veva un cognome ingom\[G]brante e rispettato
di \[C]certo in quell'ambiente da \[B]lui poco onorato.
Si \[E-]sa dove si nasce ma \[G]non come si muore
e \[C]non se un ideale ti \[B]porterà dolore.
\endverse
\beginchorus
Ma \[A-]la tua vita adesso puoi cam\[E-]biare
\[A-]solo se sei disposto a cammi\[E-]nare.
Gri\[A-]dando  forte  senza aver pa\[E-]ura 
con\[B]tando  cento passi lungo la tua strada.
Allora \[E-]uno  due  tre  quattro
\[C]cinque  dieci  cento passi,
\[G]uno  due  tre quattro  \[B]cinque dieci cento passi.
\[E-]Uno  due  tre  quattro \[C]cinque dieci cento passi,
\[G]uno  due  tre  quattro  \[B]cinque  dieci  cento passi.
\endchorus
\beginverse
Po\[E-]teva come tanti \[G]scegliere e partire
in\[C]vece lui decise di re\[B]stare.
Gli a\[E-]mici  la politica  la \[G]lotta del partito,
al\[C]le elezioni si  era candi\[B]dato.
Di\[E-]ceva da vicino li a\[G]vrebbe controllati,
ma \[C]poi non ebbe tempo perc\[B]hé venne ammazzato.
Il \[E-]nome di suo padre nella \[G]notte non è servito,
gli \[C]amici disperati non \[B]l'hanno più trovato.
\endverse
\beginchorus
Al\[A-]lora dimmi se tu sai cont\[E-]are
\[A-]dimmi se sai anche cammi\[E-]nare.
Con\[A-]tare  camminare insieme a can\[E-]tare,
la \[B]storia di Peppino e degli amici siciliani.
\[E-]Uno  due tre quattro \[C]cinque dieci cento passi.
\[G]Uno  due tre quattro \[B]cinque dieci cento passi. \rep{4}
\endchorus
\beginverse*
\[E-]Era la notte buia dello Stato Italiano \chordsoff \brk quella del nove maggio '78,
La notte di via Caetani del corpo di Aldo  Moro, \brk l'alba dei funerali di uno stato.
\endverse
\beginchorus
Al\[A-]lora dimmi se tu sai cont\[E-]are
\[A-]dimmi se sai anche cammi\[E-]nare.
Con\[A-]tare  camminare insieme a can\[E-]tare,
la \[B]storia di Peppino e degli amici siciliani.
Allora \[E-]uno due tre quattro \[C]cinque dieci cento passi,
\[G]uno due tre quattro \[B]cinque dieci cento passi!
\endchorus
\endsong
%titolo{Il vascello fantasma}
%autore{}
%album{}
%tonalita{Lam}
%famiglia{Scout}
%gruppo{}
%momenti{}
%identificatore{il_vascello_fantasma}
%data_revisione{2019_07_04}
%trascrittore{Stefano Barberini}

\beginsong{Il vascello fantasma}
\beginverse
\[A-]Pende un uomo dal pennone,
\[D-]tutt\[A-]o nero di catrame
\[D-]non \[A-]è certo un buon boccone,
\[E]per\[A-] i corvi che hanno fame.
\endverse
\beginchorus
\[A-]Pa zum! Pa pa zum!
\[E]pa \[A-]pa pa pa pa zach!
\endchorus
\chordsoff
\beginverse
Cinque teschi tutti neri
stan sul cassero di prua
son dei cinque bucanieri
che son morti alla tortura.
\endverse

\beginverse
Venti ombre tutte nere
vengon su dal boccaporto
sono venti schiavi neri
che son morti nel trasporto.
\endverse

\beginverse
Sulla tolda biancheggianti
stan tre scheletri a ballare
sono i resti dei briganti
giustiziati in alto mare.
\endverse

\beginverse
Un cadavere inchiodato
alla prua sottovento:
è un ribelle che ha pagato
con la morte il tradimento.
\endverse

\beginverse
Inchiodato al gabinetto
sta il nostromo maledetto,
che ha pagato con la vita
per averla fatta a letto.
\endverse

\beginverse
Sulla cassa posta a poppa
stan tre scheletri a giocare.
Come premio c'è una coppa
ch'era il teschio del compare.
\endverse

\beginverse
Se una notte tutta scura
sentirete un gran lamento
è la voce di Tortuga
morto in ammutinamento.
\endverse

\beginverse
Coricandovi stanotte
sentirete un sordo tonfo
è il fantasma del vascello
che reclama il suo trionfo.
\endverse

\endsong

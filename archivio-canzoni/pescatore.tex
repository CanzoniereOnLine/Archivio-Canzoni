%titolo{Pescatore}
%autore{Pierangelo Bertoli, Fiorella Mannoia}
%album{}
%tonalita{DO}
%famiglia{altre}
%gruppo{}
%momenti{}
%identificatore{pescatore}
%data_revisione{}
%trascrittore{Francesco Raccanello}
\beginsong{Pescatore}[
	by={Bertoli, Mannoia}]

\beginverse
%\newchords{Pescatore:uomo}
%\memorize{Pescatore:uomo}
\[C]Getta \[C4]le tue \[C]reti buona \[F]pesca \[G]ci sa\[C]rà
e \[C]canta le \[C4]tue can\[C]zoni che bur\[F]rasca \[C]calme\[G]rà
\[A-]pensa pensa al \[E-]tuo bambino al sa\[F]luto che \[C]ti man\[G]dò
e tua \[A-]moglie sveglia di \[E-]buon mattino con \[F]Dio di \[C]te par\[G]lò
con \[F]Dio di \[G]te par\[C]lò.
\endverse

\beginverse
%\newchords{Pescatore:donna}
%\memorize{Pescatore:donna}
\memorize
\[C]Dimmi dimmi mio signore
dimmi che torne\[G4]rà \[G]
l'\[A-]uomo mio di\[E-]fendi dal male
dai pe\[F]ricoli che trove\[G4]rà \[G]
troppo \[C]giovane son io
ed il \[D]nero è un \[G]triste \[G4]colore
la mia \[A-]pelle bianca e \[E-]profumata
ha bi\[F]sogno di ca\[C]rezze an\[G]cora
ha bi\[F]sogno di ca\[G]rezze \[C]ora
\endverse

\transpose{2}
\beginverse
^Pesca forza tira pescatore pesca e non ti fer^mare ^
^poco pesce nella ^rete
lunghi ^giorni in mezzo al m^are ^ 
^mare che non ti ha mai dato tanto 
^mare che fa bestem^miare ^
quando la ^sua furia di^venta grande 
la sua ^onda ^è un gi^gante
la sua ^onda ^è un gi^gante
\endverse

\transpose{2}
\beginverse
^Dimmi dimmi mio Signore dimmi se torne^rà ^
^l'uomo che sento ^meno mio \brk ^mentre un altro mi sorride ^già ^ 
^scaccialo dalla mia mente non in^durmi nel pec^cato ^
un ^brivido sento quan^do mi guarda \brk e una ^rosa e^gli m'ha ^dato
una ^rosa ^lui m'ha ^dato.
\endverse

\transpose{1}
\beginverse
^Rosa rossa segno d'amore 
rosa rosa dalla ^spina ^
nel si^lenzio della ^notte ora
la mia ^bocca gli è  vi^cina ^
^no mio dio non farlo tornare ^dillo tu al ^mare ^
è ^troppo forte ques^ta catena io non ^la vo^glio spez^zare
io non ^la vo^glio spez^zare.
\endverse

\transpose{-3}
\beginverse
^Pesca forza tira pescatore ^pesca e non ti fer^mare
anche ^quando l'onda ti sol^leva forte \brk e ti ^toglie dal tuo pen^sare ^
e ti ^spazza via come foglia al vento 
che vien voglia di lasciarsi an^dare ^ 
giù  leg^gero nel suo ab^braccio forte 
ma è  ^cosi cattiva ^poi la ^morte
ma è  ^cosi cattiva ^poi la ^morte.
\endverse

\transpose{2}
\beginverse
^Dimmi dimmi mio Signore dimmi se torne^rà ^
quell'^uomo che sento ^l'uomo mio \brk quell'^uomo che non sap^rà ^
^non saprà di me e di lui
delle sue pro^messe ^vane
di una ^rosa rossa qui tra ^le mie dita 
di una ^storia nata ^già fi^nita
di una ^storia nata ^già fi^nita.
\endverse

\transpose{1}
\beginverse
^Pesca forza tira pescatore ^pesca e non ti fer^mare
poco ^pesce nella ^rete
lunghi ^giorni in mezzo al ^mare ^ 
^mare che non t'ha mai dato tanto ^mare che fa bestem^miare
e si ^placa e tace ^senza resa 
e ti as^petta per ri^comin^ciare 
e ti as^petta per ri^comin^ciare.
\endverse

\endsong

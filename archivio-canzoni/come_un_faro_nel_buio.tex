%titolo{Come un faro nel buio}
%autore{Venturini, Bedocchi, Mazzanti}
%album{}
%tonalita{Mi}
%famiglia{Liturgica}
%gruppo{}
%momenti{Ingresso;Comunione;Quaresima;Penitenza,;Avvento}
%identificatore{come_un_faro_nel_buio}
%data_revisione{2011_12_31}
%trascrittore{Francesco Endrici}
\beginsong{Come un faro nel buio}[by={Venturini, Bedocchi, Mazzanti}]
\beginverse
\[C7] A \[F]Te, Signore, \[C]ci rivol\[D-]giamo, \[B&]
con tutto il \[F]cuore \[D-]noi Ti chie\[G-]diamo
di perdo\[C7]nare i nostri pec\[F]cati, \[D-]
e grazie a \[F]Te sa\[G-]remo sal\[C7]{vati\dots{}}
\endverse
\beginchorus
\[F]Come un \[C]faro nel \[F]bu\[C]io \[B&]sei per \[F]noi, Si\[G7]gno\[C7]re,
\[F]che ci il\[A-]lumina \[G7]sem\[C7]pre \[G-]nelle \[D-]notti più o\[G-]scu\[C]{re\dots{}}
\[D7]Quando \[G-]noi Ti pre\[F]ghia\[C]mo \[B&]vieni \[F]con la Tua \[G-]lu\[C7]ce
\[F]a indi\[B&]carci la \[D-]stra\[A-]da \[B&]e a do\[F]narci la \[C7]pa\[F]{ce\dots{}}
\endchorus
\beginverse 
^ La ^Tua Parola è ^fonte di ^vita ^
di veri^tà e di ^gioia infi^{nita\dots{}}
A chi la os^serva Tu sei vi^cino, ^
In ogni i^stante ^lungo il cam^{mino\dots{}}
\endverse
\beginverse
^ Sei ^Tu la sola ^via da se^guire, ^
che non dob^biamo ^mai abbando^{nare\dots{}}
Senza di ^Te noi siamo per^duti: ^
vaghiamo ^dispe^rati e smar^{riti\dots{}}
\endverse
\endsong

%titolo{Il testamento di Tito}
%autore{Fabrizio De André}
%album{}
%tonalita{SI-}
%famiglia{altre}
%gruppo{}
%momenti{}
%identificatore{il_testamento_di_tito}
%data_revisione{}
%trascrittore{Francesco Raccanello}
\beginsong{Il testamento di Tito}[
	by={De André}]
\beginverse
\[B-]Non avrai altro \[F#-]Dio all'in\[G]fuori di \[D]me, 
\[G]spesso mi ha \[A]fatto pen\[D]sare: 
\[B-]genti di\[F#-]verse \[G]venute dall'\[D]est 
\[G]dicevan che in \[A]fondo era \[D]uguale. 
\[G]Credevano a un \[A]altro di\[D]verso da \[F#-]te 
\[G]{e }non mi hanno \[A]fatto del \[D]male, 
\[G]Credevano a un \[A]altro di\[D]verso da \[F#-]te 
\[G]{e }non mi hanno \[A]fatto del \[D]male.
\endverse

\chordsoff
\beginverse
Non nominare il nome di Dio, 
non nominarlo invano. 
Con un coltello piantato nel fianco 
gridai la mia pena e il suo nome: 
ma forse era stanco, forse troppo occupato, 
e non ascoltò il mio dolore. 
Ma forse era stanco, forse troppo lontano, 
davvero lo nominai invano. 
\endverse

\beginverse
Onora il padre, onora la madre 
e onora anche il loro bastone, 
bacia la mano che ruppe il tuo naso 
perché le chiedevi un boccone: 
quando a mio padre si fermò il cuore 
non ho provato dolore. 
Quanto a mio padre si fermò il cuore 
non ho provato dolore. 
\endverse

\beginverse
Ricorda di santificare le feste. 
Facile per noi ladroni 
entrare nei templi che rigurgitan salmi 
di schiavi e dei loro padroni 
senza finire legati agli altari 
sgozzati come animali. 
Senza finire legati agli altari 
sgozzati come animali. 
\endverse

\beginverse
Il quinto dice non devi rubare 
e forse io l'ho rispettato 
vuotando, in silenzio, le tasche già gonfie 
di quelli che avevan rubato: 
ma io, senza legge, rubai in nome mio, 
quegli altri nel nome di Dio. 
Ma io, senza legge, rubai in nome mio, 
quegli altri nel nome di Dio. 
\endverse

\beginverse
Non commettere atti che non siano puri 
cioè non disperdere il seme. 
Feconda una donna ogni volta che l'ami 
così sarai uomo di fede: 
Poi la voglia svanisce e il figlio rimane 
e tanti ne uccide la fame. 
Io, forse, ho confuso il piacere e l'amore: 
ma non ho creato dolore. 
\endverse

\beginverse
Il settimo dice non ammazzare 
se del cielo vuoi essere degno. 
Guardatela oggi, questa legge di Dio, 
tre volte inchiodata nel legno: 
guardate la fine di quel nazzareno 
e un ladro non muore di meno. 
Guardate la fine di quel nazzareno 
e un ladro non muore di meno. 
\endverse

\beginverse
Non dire falsa testimonianza 
e aiutali a uccidere un uomo. 
Lo sanno a memoria il diritto divino, 
e scordano sempre il perdono: 
ho spergiurato su Dio e sul mio onore 
e no, non ne provo dolore. 
Ho spergiurato su Dio e sul mio onore 
e no, non ne provo dolore. 
\endverse

\beginverse
Non desiderare la roba degli altri 
non desiderarne la sposa. 
Ditelo a quelli, chiedetelo ai pochi 
che hanno una donna e qualcosa: 
nei letti degli altri già caldi d'amore 
non ho provato dolore. 
L'invidia di ieri non è già finita: 
stasera vi invidio la vita. 
\endverse

\beginverse
Ma adesso che viene la sera ed il buio 
mi toglie il dolore dagli occhi 
e scivola il sole al di là delle dune 
a violentare altre notti: 
io nel vedere quest'uomo che muore, 
madre, io provo dolore. 
Nella pietà che non cede al rancore, 
madre, ho imparato l'amore. 
\endverse
\endsong

%titolo{La città vecchia}
%autore{Fabrizio De André}
%album{}
%tonalita{LA-}
%famiglia{Altre}
%gruppo{}
%momenti{}
%identificatore{la_citta_vecchia}
%data_revisione{2020_03_10}
%trascrittore{Francesca Alampi}
\beginsong{La città vecchia}[by={De André}]
\ifchorded
\beginverse*
\vspace*{-0.8\versesep}
{\nolyrics \[A-]     \[D-]    \[E7]    \[A-]  }
\endverse
\fi
\beginverse
\[A-]Nei quartieri dove il \[D-]sole del buon Dio \brk \[G7]non dà i suoi \[C]raggi,
\[D-]ha già troppi impegni \[A-]per scaldar la gente \brk  \[B7]d'altri pa\[E]raggi
\[A-]una bimba canta \[D-]la canzone antic \brk a \[G7]della don\[C]naccia,
\[D-]quel che ancora non sai \[A-]tu lo imparerai \brk  \[E]solo qui fra le mie \[A-]braccia.
\endverse
\beginverse
\[A-]E se alla sua età \[D-]le difetterà \brk  \[G7]la compe\[C]tenza,
\[D-]presto affinerà \[A-]le capacità \brk  \[B7]con l'espe\[E]rienza.
\[A-]Dove sono andati i \[D-]tempi d'una volta, \brk  \[G7]per Giu\[C]none,
\[D-]quando ci voleva \[A-]per fare il mestiere \brk  \[E]anche un po' di voca\[A-]zione.
\endverse
\ifchorded
\beginverse*
\vspace*{-\versesep}
{\nolyrics \[A-] \[D-] \[G7] \[C] \[D-]     \[A-]     \[E]    \[A-]  }
\endverse
\fi
\beginverse
\[C-]Una gamba qua, \[F-]una gamba là \brk , \[B&]gonfi di \[E&]vino,
\[F-]quattro pensionati \[C-]mezzo avvelenati \brk  \[D7]al tavo\[G]lino.
\[C-]Li troverai là \[F-]col tempo che fa \brk  \[B&]estate e in\[E&]verno,
\[F-]a stratracannare, \[C-]a stramaledir le  \brk \[G]donne, il tempo \[C-]ed il governo.
\endverse
\beginverse
\[C-]Loro cercan là \[F-]la felicità \brk  \[B&]dentro a un bic\[E&]chiere,
\[F-]per dimenticare \[C-]d'esser stati presi \brk  \[D7]per il se\[G]dere.
\[C-]Ci sarà allegria, \[F-]anche in agonia, \brk  \[B&]col vino \[E&]forte,
\[F-]porterai sul viso \[C-]l'ombra di un sorriso \brk  \[G]fra le braccia \[C-]della morte.
\endverse
\beginverse 
\[A-]Vecchio professore \[D-]cosa vai cercando \brk  \[G7]in quel por\[C]tone
\[D-]forse quella che \[A-]sola ti può dare \brk  \[B7]una le\[E]zione.
\[A-]quella che di giorno \[D-]chiami con disprezzo \brk  \[G7]pubblica \[C]moglie
\[D-]quella che di notte \[A-]stabilisce il prezzo \brk  \[E]alle tue \[A-]voglie.
\endverse
\beginverse 
\[A-]Tu la cercherai, \[D-]tu la invocherai \brk  \[G7]più di una \[C]notte,
\[D-]ti alzerai disfatto \[A-]rimandando tutto \brk  \[B7]al venti\[E]sette.
\[A-]Quando incasserai, \[D-]dilapiderai \brk  \[G7]mezza pen\[C]sione,
\[D-]diecimila lire \[A-]per sentirti dire: \brk  "\[E]micio bello e bamboc\[A-]cione".
\endverse
\ifchorded
\beginverse*
\vspace*{-\versesep}
{\nolyrics \[A-]     \[D-]     \[G7]     \[C]  \[D-]     \[A-]     \[E]     \[A-]  }
\endverse
\fi
\beginverse
\[C-]Se t'inoltrerai \[F-]lungo le calate  \brk \[B&]dei vecchi \[E&]moli,
\[F-]in quell'aria spessa, \[C-]carica di sale, \brk  \[D7]gonfia di o\[G]dori:
\[C-]lì ci troverai i \[F-]ladri, gli assassini  \brk \[B&]e il tipo \[E&]strano,
\[F-]quello che ha venduto \[C-]per tremila lire \brk  \[G]sua madre ad un \[C-]nano.
\endverse
\beginverse 
\[C-]Se tu penserai \[F-]se giudicherai \brk  \[B&]da buon bor\[E&]ghese,
\[F-]li condannerai a \[C-]cinquemila anni  \brk \[D7] più le \[G]spese;
\[C-]ma se capirai, \[F-]se li cercherai \brk  \[B&]fino in \[E&]fondo
\[F-]se non sono gigli \[C-]son pur sempre figli,  \brk \[G]vittime di questo \[C-]mondo.
\endverse
\endsong
